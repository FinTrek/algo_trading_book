% !TEX root = ../../../book.tex

In Part I, our goal is to provide a review of some basic and some advanced statistical methodologies that are useful for developing trading algorithms. We begin with time series models (in Chapter 2) for univariate data and provide a broad discussion of autoregressive, moving average models---from model identification, estimation, inference and to model diagnostics. The stock price and return data do exhibit some unique features and so we identify certain stylized facts that have been confirmed; this work will greatly help to discern any anomalies as and when they arise as these anomalies generally indicate deviation from efficient market hypothesis. Wherever possible we illustrate the methodology with examples and provide codes for computing and make the data accessible to the reader. This chapter is followed by methodologies for multiple time series data and the discussion in Chapter 3 is useful from pairs trading to portfolio optimization. The last chapter (Chapter 4) in Part I contains advanced topics such as theory of point processes as trading takes place on a continual basis and understanding the market macro-structure is essential for deciding about the entry to and exit from the market. This chapter also contains other modern topics such as machine learning and artificial intelligence. 


A reader with strong statistics background can afford to skip some sections but it is recommended that the readers peruse these chapters as they contain discussion on topics that need further research work. In presenting the methodologies here, we kept the discussion lucid but immensely relevant to the main theme of this book---understanding the market behavior and developing effective trading strategies. 








