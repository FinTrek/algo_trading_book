\noindent Part~0 \par\vspace{2\baselineskip}

basics of trading are covered. The initial part of Chapter~\ref{chap:ch_trading_fund} delves into a discussion of the market structure and trading from practitioner's point of view. The terms used in this part, all can be traced back to academic literature; but the discussion is kept simple and direct. The data, which is central to all the analyses and inferences, is then introduced. The complexity of using data that can arise at irregular intervals can be better understood with an example illustrated here. Finally, the last part of this chapter contains a brief academic review of market microstructure---a topic about the mechanics of trading and how the trading can be influenced by various market designs. This is an evolving field that is of much interest to all: regulators, practitioners and academics. 



\noindent Part~1 \par\vspace{2\baselineskip}

In Part~I, our goal is to provide a review of some basic and some advanced statistical methodologies that are useful for developing trading algorithms. We begin with time series models (in Chapter~\ref{ch:ch_uvts}) for univariate data and provide a broad discussion of autoregressive, moving average models---from model identification, estimation, inference to model diagnostics. The stock price and return data exhibit some unique features and so we identify certain stylized facts regarding their behavior that have been empirically confirmed; this work will greatly help to discern any anomalies as and when they arise, as these anomalies generally indicate deviation from efficient market hypothesis. Although for modeling price and return data, only lower order models are needed, increasingly other trading features such as volume, volatility are being used in developing trading strategies. In particular,  predicting future volume flow to determine when to enter the market, requires the use of higher order models. Wherever possible we illustrate the methodology with examples and provide codes for computing, making the data accessible to the reader. We also introduce novel methodologies that have some potential for developing compelling trading models.


This chapter is followed by methodologies for multiple time series data in Chapter~\ref{ch:ch_mvts} which have applications from pairs trading to portfolio optimization. The last chapter (Chapter~\ref{ch:ch_advanced}) in Part~I contains advanced topics such as theory of point processes as trading takes place on a continual basis. Finally, this chapter also contains a very brief treatment of other modern topics such as machine learning.


A reader with a strong statistics background can afford to skip some sections but we recommend to peruse these chapters as they contain discussions on topics that need further research work. In presenting the methodologies here, we keep the discussion lucid but immensely relevant to the main theme of this book---understanding the market behavior and developing effective trading strategies. 



\noindent Part~II \par\vspace{2\baselineskip}

In this part of the book, we present trading algorithms based on statistical analyses of market data. These analyses are also guided by established principles in financial economics. The market is assumed to consist of informed and noise traders. It is also postulated that the market is efficient, and the informed traders generally gain at the expense of noise traders. Any information about a stock is quickly impounded in price by the Efficient Market Hypothesis (EMH). If there are anomalies, statistical arbitrage techniques can be employed to successfully detect and exploit them for profitable opportunities. These methodologies that were developed in the low frequency setting need to be carefully reconsidered to be successfully extended to the high frequency setting. In that space, the focus of traders has also been on Execution Strategies which are covered in Part~III.


In Chapter~\ref{ch:stat_ts}, we present some commonly used strategies based on the statistical methods described in Part~I. Although the literature is replete with numerous strategies, most of them are some variations or combinations of basic strategies described here. As price-based strategies have been fully exploited, it is important to consider strategies that use other information such as volume, market volatility, related stock performance, etc. Also presented are improved strategies based on machine learning methods. In Chapter~\ref{ch:amp}, we discuss portfolio theory---fundamental to diversification of risk and how the investment/trading strategies are handled. As portfolio rebalancing is quite common and because rebalancing requires trading, this chapter provides a comprehensive view of investment strategies with transaction costs. We conclude this part with Chapter~\ref{chap:ch_news_an} that deals with the analysis based on news and sentiment. This is essential as short-term price movements are not easily explained by the economic indicators that are slow to vary and the stock related events such as earnings announcements occur only on periodic basis. The sentiment analysis captures an understanding of how noise traders trade and can be considered as part of the emerging area of Behavioral Finance. 



\noindent Part~III \par\vspace{2\baselineskip}

In this part of the book, we cover a core topic in Algorithmic Trading: Execution Strategies. Once again, we will strive to give, from the practitioner's perspective and context, the necessary building blocks to grasp the subject and finally, the quantitative underpinnings. Our hope is that in reading this part, our readers would have enough understanding of the topic to be able to develop a basic implementation of an execution product and be able to create supporting Pre and Post-trade analytics.


We start in Chapter~\ref{chap:ch_trade_data_models} by discussing the modeling of trade data. We focus on several fundamental topics that are critical to practitioners but are not given enough attention by other treatments of the subject. Details on some of the analytics used in practice are presented. Chapter~\ref{chap:ch_mi_models} then explains one of the fundamental subjects of execution: Market Impact (MI). We bring in the intuition on the inner works of market impact and build the functional form of standard MI models from first principles. We then go into more details through a review of select literature to guide the reader that may be interested in more depth. Finally, Chapter~\ref{chap:ch_exec_models} reviews the core details of the topic by covering Execution Strategies and their components. Here again we give a practical perspective with a hope of setting the reader on a more solid understanding of the issues faced in industry.



\noindent Part~IV \par\vspace{2\baselineskip}

Algorithmic Trading is primarily a technology endeavor. The complexity of an Electronic Trading operation is often mind-boggling and revolves around the orchestration and connectivity of numerous different systems. These systems are the core of any trading operation as they are in most cases used not only to trade client flow but also do all the internal facilitation and risk trading. The subject of most of this book is to provide a toolkit for creating and for performing profitable algorithmic and execution strategies. But from the perspective of the end users, who are usually the institutional asset managers, it is paramount to have platform stability and to be able to handle more mundane processes so that they are able to trade, receive executions and do end of day reconciliations, know their positions at any point of the day. Shaving a few basis points by smart execution becomes relevant only when all the operational aspects of the trading interactions are handled flawlessly.


Designing and creating high performing algorithms is no simple matter and requires significant research work on the modeling and calibration of the strategy itself and the underlying execution. It also requires a good infrastructure: to measure and compare performance of the strategy in various settings, so that we can find and address sub-optimal behavior. Access to flexible and powerful research and developing transaction cost analysis (TCA) environments are very important aspects of the overall technology stack.

Finally, a critically important aspect of execution is to comply with the Financial Industry Regulatory Authority (FINRA) and Securities and Exchange Commission (SEC) regulatory demands for any Broker Dealer including trade, reporting, Order Audit Trial Systems (OATS), National Market Systems (NMS) rules, etc. In particular since the bizarre Flash Crash event of 2010\footnote{\url{https://en.wikipedia.org/wiki/2010_Flash_Crash}} and the spectacular trading error of Knight Capital of 2012,\footnote{\url{https://www.bloomberg.com/news/articles/2012-08-02/knight-shows-how-to-lose-440-million-in-30-minutes}} the scrutiny and regulatory burden in the space has exploded as regulators started worrying that a reckless and unchecked drive towards speed and automation could destabilize the equity markets. Today innovation and the ability to quickly bring to market new and innovative products risk being suffocated by a plethora of compliance and risk driven bureaucracy. 

This last part of the book gives a high level review of these three important, often overlooked, but critical aspects of the subject. It can only be a cursory review as the subject is immense and probably deserves books of their own.But the discussion here should provide the reader with an understanding of what is entailed in running a large trading operation.