% !TEX root = ../../../book.tex

Algorithmic Trading is primarily a technology endeavor. The complexity of an Electronic Trading operation is often mind-boggling and revolves around the orchestration and connectivity of numerous different systems. These systems are the core of any trading operation as they are in most cases used not only to trade client flow but also do all the internal facilitation and risk trading. The subject of most of this book is to provide a toolkit for creating and for performing profitable algorithmic and execution strategies. But from the perspective of the end users, who are usually the institutional asset managers, it is paramount to have platform stability and to be able to handle more mundane processes so that they are able to trade, receive executions and do end of day reconciliations, know their positions at any point of the day, etc. Shaving a few basis points by smart execution becomes relevant only when all the operational aspects of the trading interactions are handled flawlessly.


Designing and creating high performing algorithms is no simple matter and requires significant research work on the modeling and calibration of the strategy itself and the underlying execution. It also requires a good infrastructure: to measure and compare performance of the strategy in various settings, so that we can find and address sub-optimal behavior. Access to flexible and powerful research and developing transaction cost analysis (TCA) environments are very important aspects of the overall technology stack.

Finally, a critically important aspect of execution is to comply with (just citing the U.S. example here) the Financial Industry Regulatory Authority (FINRA) and Securities and Exchange Commission (SEC) regulatory demands for any Broker Dealer including trade, reporting, Order Audit Trial Systems (OATS), National Market Systems (NMS) rules, etc. In particular since the Flash Crash event of 2010\footnote{\url{https://en.wikipedia.org/wiki/2010_Flash_Crash}} and the spectacular trading error of Knight Capital of 2012,\footnote{\url{https://www.bloomberg.com/news/articles/2012-08-02/knight-shows-how-to-lose-440-million-in-30-minutes}} the scrutiny and regulatory burden in the space has exploded as regulators started worrying that a reckless and unchecked drive towards speed and automation could destabilize the equity markets.  

This last part of the book gives a high level review of these important, often overlooked, but critical aspects of the subject. It can only be a cursory review as the subject is immense and probably deserves books of their own. But the discussion here should provide the reader with an understanding of what is entailed in running a large trading operation.