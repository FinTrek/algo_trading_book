% !TEX root = ../../book.tex

Algorithmic Trading is primarily a technology endeavor. The complexity of an Electronic Trading operation is often mind-boggling and revolves around the orchestration and connectivity of numerous different systems. These systems are the core of any trading operation as it's in most cases used not only to trade client flow but also all the internal facilitation and risk trading. The subject of most of this book is the toolkit for creating profitable and performing algorithmic and execution strategies. But from the perspective of the end user, an institutional asset manager, paramount to all is platform stability and handling of all the more mundane processes for them to be able to trade, receive executions and end of day reconciliations, knowing their positions at any point of the day. Shaving a few basis points by smart execution becomes relevant only when the operational aspects of the interaction are handled flawlessly.\\

Designing and creating high performing algorithms is no simple matter and requires significant research work on the modeling and calibration of the strategy itself and the underlying quantitative models. It also requires a good infrastructure to measure and compare  performance of the strategy in various settings to find and address sub-optimal behavior. A flexible and powerful research and TCA environment is a very important aspect of the overall technology stack.\\

Finally critically important aspect of execution is to comply with the FINRA and SEC regulatory demands for any Broker Dealer including trade, reporting, OATS, NMS rules, etc. In particular since the bizarre Flash Crash event of 2010 \footnote{\url{https://en.wikipedia.org/wiki/2010_Flash_Crash}} and the spectacular trading error of Knight Capital of 2012 \footnote{\url{https://www.bloomberg.com/news/articles/2012-08-02/knight-shows-how-to-lose-440-million-in-30-minutes}} the scrutiny and regulatory burden in the space has exploded as regulators started worrying that a reckless and unchecked drive towards speed and automation could destabilized the equity markets. Today the innovation and ability to quickly bring to market new and innovative products risk being suffocated by a mountain of compliance and risk driven bureaucracy.\\

This last part of the book gives a high level review of these three important often overlooked, critical aspects of the subject. It can only be a cursory review as the subject is immense and probably deserving of several books of their own.But it should at least provide the reader with an understanding of what is entailed in running a large trading operation.

 

