% !TEX root = ../../../book.tex

In this part of the book, we present trading algorithms based on statistical analyses of market data. These analyses are also guided by established financial principles. The market is assumed to consist of informed and noise traders. It is also postulated that when the market is efficient, informed traders gain at the expense of noise traders. Any information about a stock is quickly impounded in price by the Efficient Market Hypothesis (EMH). If there are anomalies, statistical arbitrage techniques can be employed to successfully detect and exploit them for profitable opportunities. Recent evolution of electronic markets have given rise to high frequency trading based on computing technology. So the methodologies that were developed in the low frequency setting need to be carefully reconsidered. But in the high frequency setting, the focus of the traders has been on Execution Strategies which are covered in Part~III.


In Chapter~\ref{ch:stat_ts}, we present some commonly used strategies based on the statistical methods described in Part~I. Although the literature is replete with numerous strategies, most of them are some variations or combinations of basic strategies described here. As price-based strategies have been fully exploited, it is important to consider strategies that use other information such as volume, market volatility, related stock performance, etc. Also presented are improved strategies based on machine learning methods. In Chapter~\ref{ch:amp}, we discuss portfolio theory---fundamental to diversification of risk and how the investment/trading strategies are handled. As portfolio rebalancing is quite common and because rebalancing requires trading, this chapter provides a comprehensive view of investment strategies with transaction costs. We conclude this part with Chapter~\ref{chap:ch_news_an} that deals with the analysis based on news and sentiment. This is essential as short-term price movements are hard to be explained by the economic indicators that are slow to vary and the stock related events such as earnings announcements occur only on periodic basis. The sentiment analysis captures an understanding of how the noise traders trade and can be considered part of the emerging area of Behavioral Finance. 