% !TEX root = ../../book.tex
%\hfill
%\par\vspace{\baselineskip}

\chapter{The Research Stack}
Designing, testing, calibrating, measuring and improving a suite of algorithmic trading strategy is not for the faint of heart. As we have  seen It's a large scale problem that requires a deep investment and can take years to setup. One of the critical component of the overall stack and one that is often underfunded and thus underdeveloped is the research environment. For an execution business this is most often an afterthought, a setup that is scrounged together from existing piece of the infrastructure they already know the need to operate: historical market data, reference data, and historical transactional data in order to fulfill the TCA needs of the clients and a quantitative team often has to do with what is available. \\

Their needs and requirements take a back seat to the needs of the client even if in the end what the client needs is Algo performance not necessarily providing performance reports on  performance that is sub-optimal due to lack of a decent research environment. The shrewd reader will likely notice some animosity in this tirade. Must come from the years of frustration! Things though are getting better and nowadays quantitative teams a much better funded and resourced, often even with dedicated infrastructure and resources to support and maintain their research environment.\\

In this chapter we'll try to provide some pointers of what a best-in-class research stack would look like. The basic components, approaches and considerations born out two decade+  of experience and the trials and tribulations (mostly tribulations!) of yours truly. The authors wish they could dedicate the equivalent of a full book (and maybe one day they will) to this topic. Unfortunately the book is already very long and our Editor would love to see this book published before retirement so we are forced to restrain our desire and provide only a brief overview of the topic. 
 
\section{Data Infrastructure}
\section{Calibration infrastructure}
\section{Simulation Environment}
\section{TCA Environment}

