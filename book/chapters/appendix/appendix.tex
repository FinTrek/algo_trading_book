\chapter*{Appendix A: Partial R Codes}
\noindent
\begin{small}
\textbf{Chapter 2: Exercise 3}
\begin{lstlisting}
###
##FORD AGGREGATE  BY 5 MINS
ford=read.csv("ford.csv")
ford=ford[which((ford$TIME>=34200) & (ford$TIME<=57600)),]
aggprice=rep(0,dim(ford)[1])
return=rep(0,dim(ford)[1])
no.trade=rep(0,dim(ford)[1])
aggprice[1]=0
no.trade[1]=1
return[1]=0
time=ford$TIME
price=ford$PRICE
day=ford$day
starttime=time[1]

for(i in 2:dim(ford)[1])
{
if(((time[i]-starttime)<=300)&(day[i]==day[i-1])){
aggprice[i]=aggprice[i-1]+price[i]-price[i-1]
no.trade[i]=no.trade[i-1]+1}else{
aggprice[i]=0
starttime=time[i]
no.trade[i]=1
}
}
ford=cbind(ford,aggprice,no.trade)
aggford=c()
for( i in 2:dim(ford)[1])
{
if(ford$no.trade[i]==1){aggford=rbind(aggford,ford[(i-1),])}
}
dim(aggford)
day.ind=c()
for(i in 2:dim(aggford)[1]){if(aggford$day[i]!=aggford$day[i-1])
+{day.ind=c(day.ind,i)}}
fordreturn=diff(aggford$PRICE)
plot(aggford$no.trade,type="l")
abline(v=day.ind,col="red")
plot(fordreturn,type="l")
abline(v=day.ind,col="red")
acf(aggford$no.trade)
acf(fordreturn)
###
##AGGREGATE  BY 5 MINS for other stocks; follow the details for FORD above.
gm=read.csv("gm.csv")

######
#3.(d)
######
ford.M=aggford[which(aggford$weekday==2),]
ford.F=aggford[which(aggford$weekday==6),]

dur.M.ford=diff(ford.M$TIME)
dur.M.ford=dur.M.ford[dur.M.ford>0]
dur.F.ford=diff(ford.F$TIME)
dur.F.ford=dur.F.ford[dur.F.ford>0]

.
.
.

###plot duration
par(mfrow=c(2,1))
day.ind=c()
for(i in 2:dim(ford.M)[1]){if(ford.M$day[i]!=ford.M$day[i-1])
+{day.ind=c(day.ind,i)}}
plot(dur.M.ford,type="l")
abline(v=day.ind,col="red")
day.ind=c()
for(i in 2:dim(ford.F)[1]){if(ford.F$day[i]!=ford.F$day[i-1])
+{day.ind=c(day.ind,i)}}
plot(dur.F.ford,type="l")
abline(v=day.ind,col="red")

####
#3.(e)
####
library(fGarch)
garchFit(formula=~garch(1,1),data=log(dur.M.ford),cond.dist="norm")
garchFit(formula=~garch(1,1),data=log(dur.F.ford),cond.dist="norm")
.
.
.

#########
#Chapter 2: Exercise 4
#########
#step 1 calclulate volunme weighted average price
price=ford$PRICE
time=ford$TIME
size=ford$SIZE
date=ford$DATE
day=ford$day
weekday=ford$weekday
k=1
j=1
ford_unq=array(0,c(dim(ford)[1],7))
for (i in 2:dim(ford)[1])
{
	if(time[i]==time[i-1]){k=k+1}else
+{vwap=sum(price[(i-k):(i-1)]*size[(i-k):(i-1)])/sum(size[(i-k):(i-1)])
		ford_unq[j,]=c(date[(i-k)],time[(i-k)],day[(i-k)],
+weekday[(i-k)],size[(i-k)],price[(i-k)],vwap)
		k=1
		j=j+1}		
		
}
ford_unq=ford_unq[1:(j-1),]
#step 2 calculate duration
time=ford_unq[,2]
dur=diff(time)
log_dur=log(dur)
logd=0
logmean=rep(0,length(dur))
j=1

#5-minute average of log duration
for(i in 2: dim(ford_unq)[1])
{
	if(!is.na(log_dur[i-1])&(time[i]-time[j])<=300)
	{
		logd=logd+log_dur[i-1]
	}else{
		if((i-j-1)!=0){
		logmean[j:(i-1)]=logd/(i-j-1)}else{logmean[j:(i-1)]=0}
		logd=0
		j=i
	}
	if(i==dim(ford_unq)[1]){logmean[j:(i-1)]=logd/(i-j)}
}
#step3 calculate adjusted duration
addur=dur/exp(logmean)
addur[addur<0]=0
addur[addur>300]=0
day.ind=c()
for(i in 2:dim(ford\_unq)[1])\{if(ford\_unq[i,3]!=ford\_unq[(i-1),3])
+\{day.ind=c(day.ind,i)\}\}
plot(addur,type="l")
abline(v=day.ind,col="red")
#####
#4.(b)
#####
#sqrt\_addur=sqrt(addur)
garchFit(formula=~garch(1,1),data=sqrt\_addur,cond.dist="norm")
\end{lstlisting}
\newpage
\textbf{Chapter 3: Exercise 1}
\begin{lstlisting}
d_15stocks=read.csv("d_15stocks.csv")
d_indexes=read.csv("d_indexes.csv")
Permno_list=unique(d_15stocks[,1])
Ticker_list=unique(d_15stocks[,3])
stock=array(0,c(3323,15))
for(i in 1:15)
{
	stock[,i]=d_15stocks[which(d_15stocks[,1]==Permno_list[i]),4]
}
colnames(stock)=as.vector(Ticker_list)
vw=d_indexes[,2]
ew=d_indexes[,3]
#mean,sd and first order autocorrelation for 15 stocks for sample period
firstAR<-function(x)
{
	n=length(x)
	x1=x[1:(n-1)]
	x2=x[2:n]
	f.ar=cor(x1,x2)
	return(f.ar)
}
mean.stock=apply(stock,2,mean)
sd.stock=apply(stock,2,sd)
f.ar.stock=apply(stock,2,firstAR)
mean.vw=mean(vw)
sd.vw=sd(vw)
f.ar.vw=firstAR(vw)
mean.ew=mean(ew)
sd.ew=sd(ew)
f.ar.ew=firstAR(ew)
#m=c(mean.stock,mean.vw,mean.ew)
# f=c(f.ar.stock,f.ar.vw,f.ar.ew)
# s=c(sd.stock,sd.vw,sd.ew)
#for subperiod
myfun<-function(x)
{
	myfun=c(mean(x),sd(x),firstAR(x))
	return(myfun)
}
s.t1=831
s.t2=1662
s.t3=2493
s.t4=3323
stock.sub1=apply(stock[1:s.t1,],2,myfun)
stock.sub2=apply(stock[(s.t1+1):s.t2,],2,myfun)
stock.sub3=apply(stock[(s.t2+1):s.t3,],2,myfun)
stock.sub4=apply(stock[(s.t3+1):s.t4,],2,myfun)
mean.stock.sub=cbind(stock.sub1[1,],stock.sub2[1,],
+stock.sub3[1,],stock.sub4[1,])
sd.stock.sub=cbind(stock.sub1[2,],stock.sub2[2,],
+stock.sub3[2,],stock.sub4[2,])
f.ar.stock.sub=cbind(stock.sub1[3,],stock.sub2[3,],
+stock.sub3[3,],stock.sub4[3,])
##plots
lends=as.vector(Ticker_list)
matplot(t(mean.stock.sub),type="b",pch=20,xlab="Subperiods",
+ylab="Stock returns",xlim=c(1,5))
legend(locator(1),lends,lty=1:5,col=1:6,text.width=0.5)

matplot(t(sd.stock.sub),type="b",pch=20,xlab="Subperiods",
+ylab="Standard Deviation of stock returns",xlim=c(1,5))
legend(locator(1),lends,lty=1:5,col=1:6,text.width=0.5)

matplot(t(f.ar.stock.sub),type="b",pch=20,xlab="Subperiods",
+ylab="First_order AR of stock returns",xlim=c(1,5))
legend(locator(1),lends,lty=1:5,col=1:6,text.width=0.5)

i.t1=832
i.t2=1664
i.t3=2496
i.t4=3329
ind=cbind(vw,ew)
ind.sub1=apply(ind[1:i.t1,],2,myfun)
ind.sub2=apply(ind[(i.t1+1):i.t2,],2,myfun)
ind.sub3=apply(ind[(i.t2+1):i.t3,],2,myfun)
ind.sub4=apply(ind[(i.t3+1):i.t4,],2,myfun)
mean.ind.sub=cbind(ind.sub1[1,],ind.sub2[1,],ind.sub3[1,],
+ind.sub4[1,])
sd.ind.sub=cbind(ind.sub1[2,],ind.sub2[2,],ind.sub3[2,],
+ind.sub4[2,])
f.ar.ind.sub=cbind(ind.sub1[3,],ind.sub2[3,],ind.sub3[3,],
+ind.sub4[3,])
lends=c("VWRETD","EWRETD")

matplot(t(mean.ind.sub),type="b",pch=20,xlab="Subperiods",
+ylab="Means of indexes returns",xlim=c(1,5))
legend(locator(1),lends,lty=1:5,col=1:6,text.width=0.5)

matplot(t(sd.ind.sub),type="b",pch=20,xlab="Subperiods",
+ylab="Standard Deviation of indexes returns",xlim=c(1,5))
legend(locator(1),lends,lty=1:5,col=1:6,text.width=0.5)

matplot(t(f.ar.ind.sub),type="b",pch=20,xlab="Subperiods",
+ylab="First_order AR of indexes returns",xlim=c(1,5))
legend(locator(1),lends,lty=1:5,col=1:6,text.width=0.5)
#######
#Exercise 1.b
#######
norm_vw=rnorm(length(vw), mean(vw),sd(vw))
par(mfrow=c(1,2))
hist(vw,prob=T,nclass=30,xlab="VWRETD",main="")
lines(density(vw))
lines(density(norm_vw),col="red")

hist(norm_vw,prob=T,nclass=30,xlab="Normal",main="")
lines(density(norm_vw))

norm_ew=rnorm(length(ew), mean(ew),sd(ew))
par(mfrow=c(1,2))
hist(ew,prob=T,nclass=30,xlab="EWRETD",main="")
lines(density(ew))
lines(density(norm_ew),col="red")

hist(norm_ew,prob=T,nclass=30,xlab="Normal",main="")
lines(density(norm_ew))
#######
#Exercise 1.c
#######
CIfun<-function(x)
{
	n=length(x)
	xbar=mean(x,na.rm=T)
	me=qt(0.995,df=n-1)*sd(x,na.rm=T)/sqrt(n)
	CI=c(xbar-me,xbar+me)
	return(CI)
}
CI.stock=apply(stock,2,CIfun)
CI.index=apply(ind,2,CIfun)
#cbind(CI.stock,CI.index)
stock.sub1=apply(stock[1:s.t1,],2,CIfun)
stock.sub2=apply(stock[(s.t1+1):s.t2,],2,CIfun)
stock.sub3=apply(stock[(s.t2+1):s.t3,],2,CIfun)
stock.sub4=apply(stock[(s.t3+1):s.t4,],2,CIfun)
CI.stock.sub=rbind(stock.sub1,stock.sub2,stock.sub3,stock.sub4)


time=c(rep(c(1,2,3),each=831),rep(4,830))
stock_t=cbind(stock,time)
lends=c("Subperiod 1","Subperiod 2","Subperiod 3","Subperiod 4")
error.bars.by(stock_t[,1:15],time,alpha=0.01,bars=F,
+ylim=c(-0.008,0.008),main="",xlab="",ylab="Confidence
+Intervals",lty=1:5,col=1:6)
legend(locator(1),lends,lty=1:5,col=1:6)

a=cbind(c(vw[1:i.t1],NA),c(vw[(i.t1+1):i.t2],NA),
+c(vw[(i.t2+1):i.t3],NA),vw[(i.t3+1):i.t4])
error.bars(a,alpha=0.01,main="",xlab="VWRETD")
b=cbind(c(ew[1:i.t1],NA),c(ew[(i.t1+1):i.t2],NA),
+c(ew[(i.t2+1):i.t3],NA),ew[(i.t3+1):i.t4])
error.bars(b,alpha=0.01,main="",xlab="EWRETD")

#######
#Exercise 1.d
#######
library(fBasics)
SR<-function(x)
{
	sr=(max(x)-min(x))/sd(x)
	return(sr)
}
myfun2<-function(x)
{
     output=c(skewness(x),kurtosis(x), SR(x))
     return(output)
}


stock.sub1=apply(stock[1:s.t1,],2,myfun2)
stock.sub2=apply(stock[(s.t1+1):s.t2,],2,myfun2)
stock.sub3=apply(stock[(s.t2+1):s.t3,],2,myfun2)
stock.sub4=apply(stock[(s.t3+1):s.t4,],2,myfun2)
sk.stock.sub=cbind(stock.sub1[1,],stock.sub2[1,],
+stock.sub3[1,],stock.sub4[1,])
ku.stock.sub=cbind(stock.sub1[2,],stock.sub2[2,],
+stock.sub3[2,],stock.sub4[2,])
str.stock.sub=cbind(stock.sub1[3,],stock.sub2[3,],
+stock.sub3[3,],stock.sub4[3,])

lends=as.vector(Ticker_list)
matplot(t(sk.stock.sub),type="b",pch=20,xlab="Subperiods",
+ylab="Skewness of stock returns",xlim=c(1,5))
legend(locator(1),lends,lty=1:5,col=1:6,text.width=0.5)

matplot(t(ku.stock.sub),type="b",pch=20,xlab="Subperiods",
+ylab="Kurtosis of stock returns",xlim=c(1,5))
legend(locator(1),lends,lty=1:5,col=1:6,text.width=0.5)

matplot(t(str.stock.sub),type="b",pch=20,xlab="Subperiods",
+ylab="Studentized range of stock returns",xlim=c(1,5))
legend(locator(1),lends,lty=1:5,col=1:6,text.width=0.5)

#Test for Skweness (D'Agostino test) and Test for Kurtosis
(Anscombe-Glynn test)
library(moments)
apply(stock,2,agostino.test)
apply(ind,2,agostino.test)
m_15stocks=read.csv("m_15stocks.csv")
m_indexes=read.csv('m_indexes.csv')
Permno_list_m=unique(m_15stocks[,1])
Ticker_list_m=unique(m_15stocks[,3])

stock_m=array(0,c(2385,15))
for(i in 1:15)
{
	stock_m[,i]=m_15stocks[which(m_15stocks[,1]==Permno_list_m[i]),4]
}
colnames(stock_m)=as.vector(Ticker_list_m)
apply(stock_m,2,agostino.test)
apply(ind_m,2,agostino.test)
##apply Jarque-Beta test
library(tseries)
apply(stock,2,Jarque.Bera.test)
apply(ind,2,Jarque.Bera.test)
\end{lstlisting}
\textbf{Chapter 3: Exercise 2}
\begin{lstlisting}
##########
#2.(a)
##########
#library(ares)
AAPL=read.csv("d_aapl.csv")
P=rev(AAPL$Adj.Close)
n=length(P)
return=diff(P)/P[1:(n-1)]
log_return=log(1+return)
Box.test(log_return,lag=10,type="Ljung")
###########
#2.(b)
###########
ar(log_return,method="mle")
n=length(log_return)
x1=c(0,log_return[1:(n-1)])
x2=c(0,0,log_return[1:(n-2)])
x3=c(rep(0,3),log_return[1:(n-3)])
x4=c(rep(0,4),log_return[1:(n-4)])
x5=c(rep(0,5),log_return[1:(n-5)])
x6=c(rep(0,6),log_return[1:(n-6)])
ts.lm=lm(log_return~x1+x2+x3+x4+x5+x6)
summary(ts.lm)
#######
#2.(d)
#######
library(fUnitRoots)
urdfTest(log_return,lag=6)
urdfTest(log(P),lag=6)
\end{lstlisting}
\textbf{Chapter 3: Exercise 3}
\begin{lstlisting}
#####
#3.(b)
#####
library(fGarch)
garchFit(formula=~arma(6,0)+garch(1,1),data=log_return,
+cond.dist="norm")
#####
#3.(c)
#####
garchFit(formula=~arma(6,0)+garch(1,1),data=log_return,
+cond.dist="std")
#####
#3.(d)
#####
library(rugarch)
gspec=ugarchspec(variance.model = list(model = "iGARCH",
+garchOrder = c(1, 1),
submodel = NULL, external.regressors = NULL, variance.targeting = FALSE),  mean.model = list(armaOrder = c(6, 0),include.mean = TRUE, archm = FALSE,  archpow = 1, arfima = FALSE, external.regressors = NULL, archex = FALSE),  distribution.model = "norm")
ugarchfit(gspec,log_return)
\end{lstlisting}
\newpage
\textbf{Chapter 4: Exercise 5}
\begin{lstlisting}
r=read.csv("Rates.csv");
length<- nrow(r)
usd<-r$USD_by_INR
pound<-r$GBP_by_INR
euro<-r$EUR_by_INR
flip=1
LaTeX=1
sink("myLaTeXtables", append=FALSE, split=FALSE)
movingAverageSignal=function(series,ndays){
length=length(series)
signal=rep(0,length)
weights<- rep(1/ndays,ndays)
ma <- as.vector(filter(series, weights,method="convolution",side=1))
signal[series<ma]=1 # You buy
signal[series>ma]=-1 # You sell
return(signal)
}
#StartingBalance
StartingBalance=1000;
#Moving Signal to Return
convertSignalToReturn=function(series,signal,amount=100, tradingBasis=0){
l=length(series)
stock=rep(0,l); #stock: Number of shares
cash=rep(StartingBalance,l);
netLiquidAssets=rep(StartingBalance,l);
pnl=rep(0,l);
dailyReturn=rep(0,l);
for(i in 2:l){
stock[i]=stock[i-1];
if(stock[i-1]==0){ #We are open to trade
if(signal[i]==1) stock[i]<- amount/series[i] #Buy shares worth INR 100
if(signal[i]==-1) stock[i]<- -amount/series[i] #Sell shares worth INR 100
}else if((stock[i-1]>0) && (signal[i]=-1)){
#We have positive stock and we have received signal to sell
stock[i]=0
if(flip==1) stock[i]= -amount/series[i];
}else if((stock[i-1]<0) && (signal[i]=1)){
#We have negative stock and we have received signal to buy
stock[i]=0
if(flip==1) stock[i]= amount/series[i];
}
#Transaction Cost= (Difference in stock*price of stock)* tradingBasis
transactionValue=series[i]*(stock[i]-stock[i-1])
tc= abs(transactionValue)* tradingBasis
#The amount of cash = previous cash - amount spent(recd) in buying(selling) 
+ new stock- tc
cash[i]=cash[i-1] - series[i]*(stock[i]-stock[i-1]) -tc
netLiquidAssets[i]=cash[i] + series[i]* stock[i]
dailyReturn[i]=(netLiquidAssets[i]-netLiquidAssets[i-1])/netLiquidAssets[i-1]
pnl[i]= netLiquidAssets[i]-StartingBalance;
}
return(list(cash=cash, netLiquidAssets= netLiquidAssets,
+ pnl=pnl, dailyReturn=dailyReturn))
}
# Bollinger Signal (Rule 2)
bollingerSignal=function(series,ndays,nsd=2){
length=length(series)
signal=rep(0,length)
weights<- rep(1/ndays,ndays)
weights2<- rev(1:ndays)
weights2<-weights2/sum(weights2)
## and apply it as a one-sided moving average calculations, see help(filter)
bbmiddle <- as.vector(filter(series, weights,method="convolution",side=1))
bbmiddle2<- as.vector(filter(series, weights2,method="convolution",side=1))
## use var(x) = E(x^2) - E(x)^2 to compute rolling variances
v <- filter(series^2, weights, method="convolution", side=1) - bbmiddle^2
## from which we calculate rolling standard deviations the usual way
bbsd <- as.vector(sqrt(v))
bbupper <- bbmiddle2 + nsd*bbsd # upper Bollinger band
bblower <- bbmiddle2 - nsd*bbsd # lowet Bollinger band
signal[series< bblower]=1 # You buy
signal[series> bbupper]=-1 # You sell
return(signal)
}
# Momentum (Rule 1)
resistanceSupportSignal=function(series,ndays){
l=length(series)
signal=rep(0,l)
start<- ndays+1
for(i in start:l)
{
startIndex=i-ndays
endIndex=i-1
pMax=max(series[startIndex:endIndex])
pMin=min(series[startIndex:endIndex])
if(series[i]> pMax) {signal[i]=+1} #buy
if(series[i]< pMin) {signal[i]=-1} #sell
}
return(signal)
}
# Rule 4
momentumSignal=function(series,shortDays,longDays){
length=length(series)
signal=rep(0,length)
shortWeights<- rep(1/shortDays, shortDays)
shortMa <- as.vector(filter(series, shortWeights,method="convolution",side=1))
longWeights<- rep(1/longDays, longDays)
longMa <- as.vector(filter(series, longWeights,method="convolution",side=1))
start<- longDays+1
for(i in start:length)
{
#Cutting from below...
if ((shortMa[i-1]<longMa[i-1]) && (shortMa[i]>longMa[i]))
{signal[i]=-1} #Sell... Time to get out
#Cutting from above...
if ((shortMa[i-1]>longMa[i-1]) && (shortMa[i]<longMa[i]))
{signal[i]=1} #Buy... Time to get in
}
return(signal)
}
# Oscillator Rule
oscillatorSignal=function(series,ndays){
l=length(series)
signal=rep(0,l)
start<- ndays+1
for(i in start:l)
{
startIndex=i-ndays
endIndex=i-1
smallSeries=series[startIndex:endIndex]
diffSmallSeries=diff(smallSeries)
U= sum(diffSmallSeries[diffSmallSeries>0])
D= sum(diffSmallSeries[diffSmallSeries<0])
D=abs(D)
RSI=100*(U)/(U+D)
if(RSI > 70){signal[i]=-1} #You sell
if(RSI < 30){signal[i]=+1} #You buy
}
return(signal)
}
doTrade=function(signal.usd,signal.euro,signal.pound, amount, tradingBasis)
{
u= convertSignalToReturn(usd,signal.usd, amount, tradingBasis
cash.usd=u$cash;
pnl.usd=u$pnl;
netLiquidAssets.usd=u$netLiquidAssets;
dailyRet.usd=u$dailyReturn;
u= convertSignalToReturn(euro, signal.euro, amount, tradingBasis)
cash.euro=u$cash;
pnl.euro=u$pnl;
netLiquidAssets.euro=u$netLiquidAssets;
dailyRet.euro=u$dailyReturn;
u= convertSignalToReturn(pound, signal.pound, amount, tradingBasis)
cash.pound=u$cash;
pnl.pound=u$pnl;
netLiquidAssets.pound=u$netLiquidAssets;
dailyRet.pound=u$dailyReturn;
netLiquidAssets = netLiquidAssets.usd+ netLiquidAssets.euro 
+ netLiquidAssets.pound
l= length(netLiquidAssets)
portfolioReturn=rep(0,l)
for(i in 2:l)
{portfolioReturn[i]=
+(netLiquidAssets[i]-netLiquidAssets[i-1])/netLiquidAssets[i-1];}
pnl= pnl.usd+ pnl.pound+ pnl.euro
endProfit=pnl[length(pnl)]
cumulReturn <- prod(portfolioReturn +1) -1
annualizedCumulReturn <- (1 + cumulReturn)^(1/(l/365))-1
annualizedReturn <-c()
for(i in 1:l){
annualizedReturn[i] <- (1 + portfolioReturn[i])^(1/(i/365))-1
}
riskFreeRate <- .03
excessReturn <- portfolioReturn - (.03/365)
excessPortfolioReturn=rep(0,l)
for(i in 1:l){
excessPortfolioReturn[i] <- portfolioReturn[i] - .03/252
}
sharpeRatio = sqrt(252) * mean(excessPortfolioReturn)/sd(excessPortfolioReturn)
return (list(endPNL=endProfit, cumulReturn = cumulReturn,
annualizedCumulExcessReturn = annualizedCumulExcessReturn,
sharpeRatio=sharpeRatio ))
}
##########################################
# Optimality Checks
# Rule 1
#
#####################################
rule1Wrapper=function(L, amount, tradingBasis)
{
signal.usd= movingAverageSignal(usd,ndays=L)
signal.euro= movingAverageSignal(euro,ndays=L)
signal.pound= movingAverageSignal(pound,ndays=L)
ans= doTrade(signal.usd, signal.euro, signal.pound,amount, tradingBasis)
return(ans)
}
#Do a search
candidateL=c(10,20,30,40,50,100,150,200, 250, 300, 350, 400, 450, 500, 550, 600)
profits.perLag <- rep(0,length(candidateL))
annCumulExReturn <- rep(0,length(candidateL))
sharpe.perLag <- rep(0,length(candidateL))
# simpleSharpe.perLag <- rep(0,length(candidateL))
my.i <- 1
for(currentL in candidateL)
{
p=rule1Wrapper(L=currentL, amount=100,0)
profits.perLag[my.i]=p$endPNL
annCumulExReturn[my.i] =p$annualizedCumulExcessReturn
sharpe.perLag[my.i]=p$sharpeRatio
# simpleSharpe.perLag[my.i]= p$simpleSharpe
my.i <- my.i + 1
}
rule1.results.noTC <- data.frame(L = candidateL, profits =profits.perLag,
annualizedSharpe = sharpe.perLag)
if(LaTeX==1) {print(xtable(rule1.results.noTC, caption="Rule 1 Results, 
Moving Average w/ no Transaction Costs",
digits=c(1,2,4,4), display=c("d","d","f","f")))
}
if(LaTeX==0) { print(rule1.results.noTC) }
tradingBasis=0.0015
profits.perLag <- rep(0,length(candidateL))
annCumulExReturn <- rep(0,length(candidateL))
sharpe.perLag <- rep(0,length(candidateL))
my.i <- 1
for(currentL in candidateL)
{
p=rule1Wrapper(L=currentL, amount=100,tradingBasis)
profits.perLag[my.i]=p$endPNL
annCumulExReturn[my.i] =p$annualizedCumulExcessReturn
sharpe.perLag[my.i]=p$sharpeRatio
my.i <- my.i + 1
}
rule1.results.TC <- data.frame(L = candidateL, profits =profits.perLag,
annualizedSharpe = sharpe.perLag)
if(LaTeX==1){ print(xtable(rule1.results.TC, caption="Rule 1 Results, 
Moving Average with Transaction Costs",
digits=c(1,2,4,4), display=c("d","d","f","f")))
}else print(rule1.results.TC)
##########################################
#
# Rule 2
#
#####################################
rule2Wrapper=function(L, amount, tradingBasis)
{
signal.usd= bollingerSignal(usd,ndays=L)
signal.euro= bollingerSignal(euro,ndays=L)
signal.pound= bollingerSignal(pound,ndays=L)
ans= doTrade(signal.usd, signal.euro, signal.pound,amount, tradingBasis)
return(ans)
}
#Do a binary search
candidateL=c(10,20,30,40,50,100,150,200, 250, 300, 350, 400, 450, 500, 550, 600)
profits.perLag <- rep(0,length(candidateL))
annCumulExReturn <- rep(0,length(candidateL))
sharpe.perLag <- rep(0,length(candidateL))
my.i <- 1
for(currentL in candidateL)
{
p= rule2Wrapper(L=currentL, amount=100,0)
profits.perLag[my.i]=p$endPNL
annCumulExReturn[my.i] =p$annualizedCumulExcessReturn
sharpe.perLag[my.i]=p$sharpeRatio
my.i <- my.i + 1
}
rule2.results.noTC <- data.frame(L = candidateL, profits =profits.perLag,
annualizedSharpe = sharpe.perLag)
if(LaTeX==1){ print(xtable(rule2.results.noTC, caption="Rule 2 Results, 
Bollinger Bands with no Transaction Costs",
digits=c(1,2,4,4), display=c("d","d","f","f")))
}else print(rule2.results.noTC)
tradingBasis=0.0015
profits.perLag <- rep(0,length(candidateL))
annCumulExReturn <- rep(0,length(candidateL))
sharpe.perLag <- rep(0,length(candidateL))
my.i <- 1
for(currentL in candidateL)
{
p=rule2Wrapper(L=currentL, amount=100,tradingBasis)
profits.perLag[my.i]=p$endPNL
annCumulExReturn[my.i] =p$annualizedCumulExcessReturn
sharpe.perLag[my.i]=p$sharpeRatio
my.i <- my.i + 1
}
rule2.results.TC <- data.frame(L = candidateL, profits =profits.perLag,
annualizedSharpe = sharpe.perLag)
if(LaTeX==1){ print(xtable(rule2.results.TC, caption="Rule 2 Results, 
Bollinger Bands with Transaction Costs",
digits=c(1,2,4,4), display=c("d","d","f","f")))
} else print(rule2.results.TC)
##########################################
#
# Rule 3
#
#####################################
rule3Wrapper=function(L, amount, tradingBasis)
{
signal.usd= resistanceSupportSignal(usd,ndays=L)
signal.euro= resistanceSupportSignal(euro,ndays=L)
signal.pound= resistanceSupportSignal(pound,ndays=L)
ans= doTrade(signal.usd, signal.euro, signal.pound, amount, tradingBasis)
return(ans)
}
#Do a binary search
candidateL=c(10,20,30,40,50,100,150,200, 250, 300, 350, 400, 450, 500, 550, 600)
profits.perLag <- rep(0,length(candidateL))
annCumulExReturn <- rep(0,length(candidateL))
sharpe.perLag <- rep(0,length(candidateL))
my.i <- 1
for(currentL in candidateL)
{
p= rule3Wrapper(L=currentL, amount=100,0)
profits.perLag[my.i]=p$endPNL
annCumulExReturn[my.i] =p$annualizedCumulExcessReturn
sharpe.perLag[my.i]=p$sharpeRatio
my.i <- my.i + 1
}
rule3.results.noTC <- data.frame(L = candidateL, profits =profits.perLag,
annualizedSharpe = sharpe.perLag)
if(LaTeX==1){ print(xtable(rule3.results.noTC, caption="Rule 3 Results, 
Resistance-Support w/ no Transaction Costs",
digits=c(1,2,4,4), display=c("d","d","f","f")))
} else {print(rule3.results.noTC)}
tradingBasis=0.0015
profits.perLag <- rep(0,length(candidateL))
annCumulExReturn <- rep(0,length(candidateL))
sharpe.perLag <- rep(0,length(candidateL))
my.i <- 1
for(currentL in candidateL)
{
p=rule3Wrapper(L=currentL, amount=100,tradingBasis)
profits.perLag[my.i]=p$endPNL
annCumulExReturn[my.i] =p$annualizedCumulExcessReturn
sharpe.perLag[my.i]=p$sharpeRatio
my.i <- my.i + 1
}
rule3.results.TC <- data.frame(L = candidateL, profits =profits.perLag,
annualizedSharpe = sharpe.perLag)
if(LaTeX==1) {print(xtable(rule3.results.TC, caption="Rule 3 Results, 
Resistance-Support with Transaction Costs",
digits=c(1,2,4,4), display=c("d","d","f","f")))
} else print(rule3.results.TC)
##########################################
#
# Rule 4
#
#####################################
rule4Wrapper=function(shortDays, longDays, amount, tradingBasis)
{
signal.usd= momentumSignal(usd,shortDays= shortDays, longDays= longDays)
signal.euro= momentumSignal(euro,shortDays= shortDays, longDays= longDays)
signal.pound= momentumSignal(pound,shortDays= shortDays, longDays= longDays)
ans= doTrade(signal.usd, signal.euro, signal.pound, amount, tradingBasis)
return(ans)
}
#Do a binary search
candidateLongDays=c(10,20,40,80,160,320,640)
for(currentLongDay in candidateLongDays)
{
candidateShortDays=seq(from=5, to=currentLongDay-1, by=5)
for(currentShortDay in candidateShortDays)
{
p= rule4Wrapper(shortDays= currentShortDay, longDays= currentLongDay, 
amount=100,0)
endPNL=p$endPNL
sharpeRatio=p$sharpeRatio
# print(sprintf("LongDay=%f, ShortDay=%f, endPNL=%0.4f,sharpeRatio=%0.4f", 
currentLongDay, currentShortDay, endPNL,sharpeRatio))
}
}
#Do a binary search
candidateLongDays=c(10,20,40,80,160,320,640)
for(currentLongDay in candidateLongDays)
{
candidateShortDays=seq(from=5, to=currentLongDay-1, by=5)
for(currentShortDay in candidateShortDays)
{
p= rule4Wrapper(shortDays= currentShortDay, longDays= currentLongDay, 
amount=100, .0015)
endPNL=p$endPNL
sharpeRatio=p$sharpeRatio
# print(sprintf("LongDay=%f, ShortDay=%f, endPNL=%0.4f,sharpeRatio=%0.4f", 
currentLongDay, currentShortDay, endPNL,sharpeRatio))
}
}
##########################################
#
# Partb- Oscillator Rule
#
#####################################
partBWrapper=function(L, amount, tradingBasis)
{
signal.usd= oscillatorSignal(usd,ndays=L)
signal.euro= oscillatorSignal(euro,ndays=L)
signal.pound= oscillatorSignal(pound,ndays=L)
ans= doTrade(signal.usd, signal.euro, signal.pound, amount, tradingBasis)
return(ans)
}
#Do a binary search
candidateL=c(10,20,30,40,50,100,150,200, 250, 300, 350, 400, 450, 500, 550, 600)
profits.perLag <- rep(0,length(candidateL))
annCumulExReturn <- rep(0,length(candidateL))
sharpe.perLag <- rep(0,length(candidateL))
my.i <- 1
for(currentL in candidateL)
{
p= partBWrapper(L=currentL, amount=100,0)
profits.perLag[my.i]=p$endPNL
annCumulExReturn[my.i] =p$annualizedCumulExcessReturn
sharpe.perLag[my.i]=p$sharpeRatio
my.i <- my.i + 1
}
Oscillator.results.noTC <- data.frame(L = candidateL, profits =profits.perLag,
annualizedSharpe = sharpe.perLag)
if(LaTeX==1){ print(xtable(Oscillator.results.noTC,
caption="Oscillator Rule Results, no Transaction Costs",
digits=c(1,2,4,4), display=c("d","d","f","f")))
} else print(Oscillator.results.noTC)
tradingBasis=0.0015
profits.perLag <- rep(0,length(candidateL))
annCumulExReturn <- rep(0,length(candidateL))
sharpe.perLag <- rep(0,length(candidateL))
my.i <- 1
for(currentL in candidateL)
{
p=partBWrapper(L=currentL, amount=100,tradingBasis)
profits.perLag[my.i]=p$endPNL
annCumulExReturn[my.i] =p$annualizedCumulExcessReturn
sharpe.perLag[my.i]=p$sharpeRatio
my.i <- my.i + 1
}
Oscillator.results.TC <- data.frame(L = candidateL, profits =profits.perLag,
sharpe = sharpe.perLag)
if(LaTeX==1) {print(xtable(Oscillator.results.TC, caption="Oscillator Rule 
Results with Transaction Costs",
digits=c(1,2,4,4), display=c("d","d","f","f")))
} else print(Oscillator.results.TC)
\end{lstlisting}
\textbf{Chapter 4: Exercise 5b}
\begin{lstlisting}
r=read.csv("~/Documents/Personal/Stats242/Hw/Hw2/Rates.csv")
length<- nrow(r)
usd<-r$USD_by_INR
pound<-r$GBP_by_INR
euro<-r$EUR_by_INR
#This function returns end of month indexes of end of month
getEndOfMonthIndices=function(dateSeries){
l=length(dateSeries)
dateT=as.character(dateSeries)
d=as.Date(dateT,"%Y%m%d")
m=months(d)
endOfMonth=rep(-1,l)
curMonth=1;
for(i in 2:l)
{
if(m[i]!=m[i-1])
{endOfMonth[i-1]=curMonth; curMonth=curMonth+1}
}
return((list(months=m, endOfMonth=endOfMonth)))
}
#
e= getEndOfMonthIndices(r$Date)
endOfMonthIndex= e$endOfMonth
getLast3MonthsOfdata=function(curMonth){
curIndex= which(endOfMonthIndex==curMonth)
startMonth=curMonth-3
startIndex= which(endOfMonthIndex==startMonth)+1
if(startMonth==0){startIndex=1}
usd.3mths<-usd[startIndex:curIndex]
pound.3mths<-pound[startIndex:curIndex]
euro.3mths<-euro[startIndex:curIndex]
#Now we normalize this series
usd.3mths.normalize=usd.3mths/usd.3mths[1]
pound.3mths.normalize= pound.3mths/pound.3mths[1]
euro.3mths.normalize= euro.3mths/euro.3mths[1]
return(list(usd.3mths= usd.3mths.normalize, pound.3mths=
pound.3mths.normalize, euro.3mths= euro.3mths.normalize,usd.const=usd.}
#Take two normalized series and find the difference and variance
getPriceDifference=function(series1, series2){
start1<-1
end1<-length(series1)
t1=length(series1)
pd=sum((series1[start1:end1]-series2[start1:end1])^2)/t1
x1=((series1[start1:end1]-series2[start1:end1])^2-pd)^2
x2=sum(x1)/(t1-1)
stdPD=sqrt(x2)
return(list(pd= pd, stdPD = stdPD))
}
getNext3MonthsOfdata =function(curMonth,u){
curIndex= which(endOfMonthIndex==curMonth)+1
endMonth=curMonth+ eligibilityPeriod
if(endMonth>47){endMonth =47}
endIndex= which(endOfMonthIndex== endMonth)
usd.3mths<-usd[curIndex:endIndex]
pound.3mths<-pound[curIndex:endIndex]
euro.3mths<-euro[curIndex:endIndex]
#Now we normalize this series
usd.3mths.normalize=usd.3mths/u$usd.const
pound.3mths.normalize= pound.3mths/u$pound.const
euro.3mths.normalize= euro.3mths/u$euro.const
return(list(usd.3mths= usd.3mths.normalize, pound.3mths=
pound.3mths.normalize, euro.3mths= euro.3mths.normalize,usd.orig= usd.}
###############
#
#
# The Main function that runs simulation
#
#
####################
StartingBalance=1000;
KTradingBasis=0;
Amount=100
runSimulation =function(name,series1,series2,stdPD,series1.Orig, series2.Orig){
trade=list();
trade$name=name;
trade$series1=series1;
trade$series2=series2;
l=length(series1)
series1Shares=rep(0,l)
series2Shares=rep(0,l)
cash=rep(StartingBalance,l)
netLiquidAssets=rep(StartingBalance,l);
pnl=rep(0,l)
dailyReturn=rep(0,l)
amount=Amount
#Internal to trade...
biggerSeries=0
tradeTime =0; #
maxTime=l;
#This is eligibilityPeriod + Convergence Period
if(eligibilityPeriod>1)
{maxTime=l-21;} #Reserve last 21 days for convergence period only
for(i in 2:l)
{
series1Shares[i]= series1Shares[i-1]
series2Shares[i]= series2Shares[i-1]
if (tradeTime==0 && ((series1[i]-series2[i])^2>2*stdPD) && (i<=maxTime))
{
tradeTime=i; #We are now in trade
print(sprintf("TradeStart=%f",tradeTime))
#Series 1 is higher than series 2; buy series2 and sell series 1
if(series1[i]>series2[i])
{
series1Shares[i]= -amount/series1.Orig[i]; #Use non-normalized
series2Shares[i]= amount/series2.Orig[i]; #Use non-normalized
biggerSeries=1;
}
if(series1[i]<series2[i])
{
series1Shares[i]= amount/series1.Orig[i]; #Use non-normalized
series2Shares[i]= -amount/series2.Orig[i]; #Use non-normalized
biggerSeries =2;
}
}
#Check if pair has converged...
if(tradeTime >0)
{
resetVariables=0
if(biggerSeries==1 && (series2[i]> series1[i]))
{resetVariables=1;}
if(biggerSeries==2 && (series1[i]> series2[i]))
{resetVariables=2;}
#if(i-tradeTime>21)
# {resetVariables=3;}
#Reset all variables
if(resetVariables>0)
{
print(sprintf("TradeEnd=%f,Reason=%f, pnl=%0.4f",i, resetVariables,pnl[i-1]))
series1Shares[i]=0;
series2Shares[i]=0;
biggerSeries=0;
tradeTime=0
}
}
#Or if i==l --> just close the trade
if(i==l && tradeTime>0)
{
print(sprintf("TradeEnd=%f,Reason=NoConvergence, pnl=%0.4f",i, pnl[i-1]))
series1Shares[i]=0;
series2Shares[i]=0
}
#Transaction Cost= (Difference in stock*price of stock)* tradingBasis
tv1= series1.Orig[i]*(series1Shares[i]-series1Shares[i-1])
tv2= series2.Orig[i]*(series2Shares[i]-series2Shares[i-1])
tc= (abs(tv1)+abs(tv2))* KTradingBasis
#Now calculate the cash in hand...
cash[i]=cash[i-1]-series1.Orig[i]*(series1Shares[i]-series1Shares[i-1])-
series2.Orig[i]*(series2Shares[i]-series2Shares[#Calculate NetLiquidAssets in hand
netLiquidAssets[i]=cash[i] + series1.Orig[i]* series1Shares[i] + series2.Orig[i]* 
series2Shares[i]
dailyReturn[i]=(netLiquidAssets[i]-netLiquidAssets[i-1])/netLiquidAssets[i-1]
pnl[i]= netLiquidAssets[i]-StartingBalance;
}# end
trade$series1Shares= series1Shares
trade$series2Shares= series2Shares
trade$netLiquidAssets= netLiquidAssets
trade$dailyReturn = dailyReturn
trade$portfolioReturn = dailyReturn
trade$cash=cash
trade$pnl=pnl
return(trade)
}
##########
#
# Let's do trading...
#
###########
runTrade=function(u,v){
usdAndPound=getPriceDifference(u$usd.3mths, u$pound.3mths)
usdAndEuro=getPriceDifference(u$usd.3mths, u$euro.3mths)
poundAndEuro=getPriceDifference(u$pound.3mths, u$euro.3mths)
minDist=min(usdAndPound$pd, usdAndEuro$pd, poundAndEuro$pd)
minTrade=1
if(usdAndEuro$pd ==minDist){minTrade = 2}
if(poundAndEuro$pd ==minDist){minTrade=3}
if(minTrade==1)
{
print ("USD And Pound")
trade=runSimulation(name="usdAndPound",v$usd.3mths,v$pound.3mths, usdAndPound$stdPD, v$usd.orig, v$pound.orig)
}
if(minTrade==2)
{
print ("USD And Euro")
trade= runSimulation(name="usdAndEuro",v$usd.3mths,v$euro.3mths, usdAndEuro$stdPD, v$usd.orig, v$euro.orig)
}
if(minTrade==3)
{
print ("Pound And Euro")
trade= runSimulation(name="poundAndEuro",v$pound.3mths,v$euro.3mths, poundAndEuro$stdPD, v$pound.orig, v$euro.orig)
}
return(trade)
}
# Now let's start pairs trading for each month...
eligibilityPeriod=3; #1 month eligibility
KTradingBasis=0.0015;
getResults=function()
{
totalNumberOfMonths=max(endOfMonthIndex)
monthsAvailableForTrading= totalNumberOfMonths-eligibilityPeriod
profit=0; pr=0;
curMonth=3;
PNL=0;
PortfolioReturns=0;
while(curMonth < monthsAvailableForTrading){
print(sprintf(">CurMonth=%0.0f, pr=%0.4f, profit=%0.4f",curMonth,pr, profit))
u=getLast3MonthsOfdata(curMonth)
v=getNext3MonthsOfdata(curMonth,u)
trade=runTrade(u,v)
lastPNL=PNL[length(PNL)];
thisPNL=trade$pnl
overallPNL= lastPNL+ thisPNL
PNL=c(PNL, overallPNL)
PortfolioReturns=c(PortfolioReturns,trade$dailyReturn)
pr=trade$pnl[length(trade$pnl)]
profit=profit+pr
print(sprintf("<CurMonth=%0.0f, pr=%0.4f, profit=%0.4f",curMonth,pr, profit))
curMonth=curMonth+eligibilityPeriod
}
sharpeRatio=(mean(PortfolioReturns)-0.03/252)/sd(PortfolioReturns)
sharpeRatio.Ananualized= sharpeRatio*sqrt(252)
print(sprintf("Profit=%0.4f, SharpeRatio=%0.4f, sharpeRatio.Ananualized=%0.4f", profit, sharpeRatio, sharpeRatio.Ananualized))
if(eligibilityPeriod>1)
title="Eligibility Period=2 + additional 1 month for convergence" else
title="Eligibility Period=1"
if(KTradingBasis==0)
title=paste (title, "No Fees") else
title=paste (title, "With Fees")
par(mfrow=c(2,1))
par(oma=c(0,0,2,0))
plot(PortfolioReturns, main="Portfolio Return",type="l")
plot(PNL, main="PNL",type="l")
mtext(title, outer = T, cex = 1)
}
# Now, let's start trading
eligibilityPeriod=1; KTradingBasis=0;
getResults();
eligibilityPeriod=3; KTradingBasis=0;
getResults();
eligibilityPeriod=1; KTradingBasis=0.0015;
getResults();
eligibilityPeriod=3; KTradingBasis=0.0015;
getResults();
\end{lstlisting}
\textbf{Chapter 4: Exercise 5c}
\begin{lstlisting}
library('tseries')
library('urca')
library('vars')
r=read.csv("~/Documents/Personal/Stats242/Hw/Hw2/Rates.csv")
# Co-Integration
# -Use the log-price as recommended in Vidyamurthy (2004)
# -
length<- nrow(r)
usdInitial=r$USD_by_INR
poundInitial=r$GBP_by_INR
euroInitial=r$EUR_by_INR
usd<-log(r$USD_by_INR)
pound<-log(r$GBP_by_INR)
euro<-log(r$EUR_by_INR)
#This function returns end of month indexes of end of month
getEndOfMonthIndices=function(dateSeries){
l=length(dateSeries)
dateT=as.character(dateSeries)
d=as.Date(dateT,"%Y%m%d")
m=months(d)
endOfMonth=rep(-1,l)
curMonth=1;
for(i in 2:l)
{
if(m[i]!=m[i-1])
{endOfMonth[i-1]=curMonth; curMonth=curMonth+1}
}
return((list(months=m, endOfMonth=endOfMonth)))
}
#
e= getEndOfMonthIndices(r$Date)
endOfMonthIndex= e$endOfMonth
getLast3MonthsOfdata=function(curMonth){
curIndex= which(endOfMonthIndex==curMonth)
startMonth=curMonth-3
startIndex= which(endOfMonthIndex==startMonth)+1
if(startMonth==0){startIndex=1}
usd.3mths<-usd[startIndex:curIndex]
pound.3mths<-pound[startIndex:curIndex]
euro.3mths<-euro[startIndex:curIndex]
return(list(usd.3mths= usd.3mths, pound.3mths= pound.3mths, euro.3mths= euro.3mths ))
}
getNext3MonthsOfdata =function(curMonth,u){
curIndex= which(endOfMonthIndex==curMonth)
endMonth=curMonth+ eligibilityPeriod
if(endMonth>47){endMonth =47}
endIndex= which(endOfMonthIndex== endMonth)
usd.3mths<-usd[curIndex:endIndex]
pound.3mths<-pound[curIndex:endIndex]
euro.3mths<-euro[curIndex:endIndex]
#Now we normalize this series
usd.orig<-usdInitial[curIndex:endIndex]
pound.orig<-poundInitial[curIndex:endIndex]
euro.orig<-euroInitial[curIndex:endIndex]
return(list(usd.3mths= usd.3mths, pound.3mths= pound.3mths, euro.3mths= euro.3mths,usd.orig= usd.orig, pound.orig= pound.orig,
}
isCointegrated=function(s1,s2){
m<-lm(s1~s2)
mu<-coef(m)[1]
beta <- coef(m)[2]
stdev <-sd(m$resid)
sprd <- m$resid
#use adf.test Standarad ADF.Test test instead of Au
ht <- adf.test(sprd, alternative="stationary", k=0)
if (ht$p.value < 0.05) {
^E
^E
^E cat("***The spread is likely mean-reverting****\n")
} else {
#cat("Not Mean-Reverting\n")
^E
^E
^E beta<-0
}
return(list(mu=mu, beta=beta,stdev= stdev))
}
multipleCointegrated=function(s1,s2,s3){
m<-lm(s1~s2+s3)
mu<-coef(m)[1]
beta1 <- coef(m)[2]
beta2 <- coef(m)[3]
stdev <-sd(m$resid)
sprd <- m$resid
#use adf.test
ht <- adf.test(sprd, alternative="stationary", k=0)
if (ht$p.value < 0.05) {
^E
^E
^E cat("+++++The spread is likely mean-reverting++++\n")
} else {
#cat("-Not reverting-\n")
^E
^E
^E beta1<-0
beta2<-0
}
return(list(mu=mu, beta1=beta1, beta2=beta2, stdev= stdev))
}
###############
#
#
# The Main function that runs simulation
#
#
####################
#StartingBalance
StartingBalance=1000;
KTradingBasis=0;
Amount=100
runSimulation=function(name, series1, series2, series3=series2, beta, beta2=0, mu, stdev, series1.Orig, series2.Orig, series3.Orig= series2.{
trade=list();
l=length(series1)
beta1=beta
trade$name=name;
trade$beta1=beta1;
trade$beta2=beta2;
trade$mu=mu;
trade$stdev=stdev;
trade$series1=series1;
trade$series2=series2;
trade$series3=series3;
#This is series1 shares
series1Shares= rep(0,l);
series2Shares= rep(0,l);
series3Shares= rep(0,l);
cash=rep(0,l);
netLiquidAssets=rep(0,l);
pnl=rep(0,l);
dailyReturn=rep(0,l)
returns=rep(0,l)
#Plain exit from here...
if(beta1==0)
{
trade$series1Shares=series1Shares;
trade$series2Shares=series2Shares;
trade$series3Shares=series3Shares;
trade$dailyReturn = dailyReturn
trade$cash=cash
trade$pnl=pnl
trade$netLiquidAssets= netLiquidAssets
return(trade)
}
cash=rep(StartingBalance,l);
netLiquidAssets=rep(StartingBalance,l);
print(name)
############
#
# Now we run simulation...
# -Series were co-integrated, so we run simulation here...
#
##############
delta=2*stdev;
amount=Amount
tradeTime=0;
biggerSeries=0;
maxTime=l;
#This is eligibilityPeriod + Convergence Period
if(eligibilityPeriod>1)
{maxTime=l-21;} #Reserve last 21 days for convergence period only
for(i in 2:l)
{
series1Shares[i]= series1Shares[i-1]
series2Shares[i]= series2Shares[i-1]
series3Shares[i]= series3Shares[i-1]
#We buy a share of one stock and short a share of gamma stock if
# w_t = p1_t- \beta p2_t > mu_w-Delta
#Get Price Difference
pd= series1[i]-beta1*series2[i]-beta2*series3[i] -mu;
if (tradeTime==0 && (abs(pd)>delta ) && (i<=maxTime))
{
tradeTime=i; #We are now in trade
print(sprintf("TradeStart=%f",tradeTime))
#Series 1 is higher than series 2;
# -Sell a share of 1st stock and buy beta share of 2nd stock...
proportion= amount/(beta1*series2.Orig[i]+beta2*series2.Orig[i]);
if(pd<0)
{
series1Shares[i]= amount/series1.Orig[i]; #Use non-normalized
series2Shares[i]= -beta1*proportion; #Use non-normalized
series3Shares[i]= -beta2*proportion; #Use non-normalized
biggerSeries =2;
}
if(pd>0)
{
series1Shares[i]= -amount/series1.Orig[i]; #Use non-normalized
series2Shares[i]= beta1*proportion; #Use non-normalized
series3Shares[i]= beta2*proportion; #Use non-normalized
biggerSeries=1;
}
}
#Check if pair has converged...
if(tradeTime >0)
{
resetVariables=0
if(biggerSeries==1 && (pd<0))
{resetVariables=1;}
if(biggerSeries==2 && (pd>0))
{resetVariables=2;}
#Reset all variables
if(resetVariables>0)
{
print(sprintf("TradeEnd=%f,Reason=%f, pnl=%0.4f",i, resetVariables,pnl[i-1]))
series1Shares[i]=0;
series2Shares[i]=0;
series3Shares[i]=0;
biggerSeries=0;
tradeTime=0
}
}
#Or if i==l --> just close the trade
if(i==l)
{
print(sprintf("TradeEnd=%f,Reason=NoConvergence, pnl=%0.4f",i, pnl[i-1]))
series1Shares[i]=0;
series2Shares[i]=0;
series3Shares[i]=0;
biggerSeries=0;
tradeTime=0
}
#Transaction Cost= (Difference in stock*price of stock)* tradingBasis
tv1= series1.Orig[i]*(series1Shares[i]-series1Shares[i-1])
tv2= series2.Orig[i]*(series2Shares[i]-series2Shares[i-1])
tv3= series3.Orig[i]*(series3Shares[i]-series3Shares[i-1])
tc= (abs(tv1)+abs(tv2)+abs(tv3))* KTradingBasis
#Now calculate the cash in hand...
cash[i]=cash[i-1]-series1.Orig[i]*(series1Shares[i]-series1Shares[i-1])-series2.Orig[i]*(series2Shares[i]-series2Shares[#Calculate PNL in hand
netLiquidAssets[i]=cash[i] + series1.Orig[i]* series1Shares[i] + series2.Orig[i]* series2Shares[i]+ series3.Orig[i]*series3Shares[dailyReturn[i]=(netLiquidAssets[i]-netLiquidAssets[i-1])/netLiquidAssets[i-1]
pnl[i]= netLiquidAssets[i]-StartingBalance;
}#end of simulation...
trade$series1Shares= series1Shares
trade$series2Shares= series2Shares
trade$series3Shares= series3Shares
trade$netLiquidAssets= netLiquidAssets
trade$dailyReturn = dailyReturn
trade$cash=cash
trade$pnl=pnl
return(trade)
}
runTrade=function(u,v)
{
#For USD And Pound
x1= isCointegrated(u$usd.3mths,u$pound.3mths);
trade.usaAndPound= runSimulation(name="usaAndPound",series1=v$usd.3mths,series2=v$pound.3mths,
beta=x1$beta, mu=x1$mu,
stdev= x1$stdev,
series1.Orig=v$usd.orig,
series2.Orig=v$pound.orig
)
x2= isCointegrated(u$pound.3mths,u$euro.3mths);
trade.poundAndEuro= runSimulation(name="poundAndEuro",series1=v$pound.3mths,series2=v$euro.3mths,
beta=x2$beta, mu=x2$mu,
stdev= x2$stdev,
series1.Orig=v$pound.orig,
series2.Orig=v$euro.orig
)
x3= isCointegrated(u$usd.3mths,u$euro.3mths);
trade.usaAndEuro= runSimulation(name="usaAndEuro",series1=v$usd.3mths,series2=v$euro.3mths,
beta=x3$beta, mu=x3$mu,
stdev= x3$stdev,
series1.Orig=v$usd.orig,
series2.Orig=v$euro.orig
)
x4= multipleCointegrated(u$usd.3mths,u$pound.3mths,u$euro.3mths);
trade.all= runSimulation(name="AllThree",series1=v$usd.3mths,series2=v$pound.3mths,series3=v$euro.3mths,
beta=x4$beta1, beta2=x4$beta2, mu=x4$mu,
stdev= x4$stdev,
series1.Orig=v$usd.orig,
series2.Orig=v$pound.orig,
series3.Orig=v$euro.orig
)
#Overall Profit and Loss
pnl=rep(0,length(v$usd.3mths))
portfolioReturns=rep(0,length(pnl))
netLiquidAssets=rep(0,length(pnl))
pnl= trade.usaAndPound$pnl+ trade.poundAndEuro$pnl+ trade.usaAndEuro$pnl
netLiquidAssets= trade.usaAndPound$netLiquidAssets + trade.poundAndEuro$netLiquidAssets + trade.usaAndEuro$netLiquidAssets
if(KSupportThreeSeries==1)
{
pnl=pnl+ trade.all$pnl
netLiquidAssets= netLiquidAssets+trade.all$netLiquidAssets
}
for(i in 2:length(netLiquidAssets))
{
portfolioReturns[i]= (netLiquidAssets[i]-netLiquidAssets[i-1])/netLiquidAssets[i-1]
}
tradeHappened=0;
if(x1$beta!=0 || x2$beta!=0|| x3$beta!=0 )
{tradeHappened=1}
if(x4$beta1!=0 && KSupportThreeSeries==1 )
{tradeHappened=1}
return(list(pnl=pnl, netLiquidAssets= netLiquidAssets, tradeHappened= tradeHappened, portfolioReturns= portfolioReturns ))
}
#############################################################################
#
#
#
#
#
#
#
#############################################################################
# Configuration Parameters
eligibilityPeriod=3
KTradingBasis=0.0015;
KSupportThreeSeries=1;
getResults=function()
{
totalNumberOfMonths=max(endOfMonthIndex)
monthsAvailableForTrading= totalNumberOfMonths-eligibilityPeriod
profit=0;
pr=0;
curMonth=3;
PNL=0;
PortfolioReturns=0;
while(curMonth < monthsAvailableForTrading){
print(sprintf(">CurMonth=%0.0f, pr=%0.4f, profit=%0.4f",curMonth,pr, profit))
u=getLast3MonthsOfdata(curMonth)
v=getNext3MonthsOfdata(curMonth,u)
trade=runTrade(u,v)
if(trade$tradeHappened==1)
{
lastPNL=PNL[length(PNL)];
thisPNL=trade$pnl
overallPNL= lastPNL+ thisPNL
PNL=c(PNL, overallPNL)
PortfolioReturns=c(PortfolioReturns,trade$portfolioReturns)
pr=trade$pnl[length(trade$pnl)]
profit=profit+pr
}
print(sprintf("<CurMonth=%0.0f, pr=%0.4f, profit=%0.4f",curMonth,pr, profit))
curMonth=ifelse(trade$tradeHappened==0,curMonth+1,curMonth+eligibilityPeriod)
}
sharpeRatio=(mean(PortfolioReturns)-0.03/252)/sd(PortfolioReturns)
sharpeRatio.Ananualized= sharpeRatio*sqrt(252)
print(sprintf("Profit=%0.4f, SharpeRatio=%0.4f, sharpeRatio.Ananualized=%0.4f", profit, sharpeRatio, sharpeRatio.Ananualized))
}
KSupportThreeSeries=0
# Now, let's start trading
eligibilityPeriod=1; KTradingBasis=0;
getResults();
eligibilityPeriod=3; KTradingBasis=0;
getResults();
eligibilityPeriod=1; KTradingBasis=0.0015;
getResults();
eligibilityPeriod=3; KTradingBasis=0.0015;
getResults();
\end{lstlisting}
\textbf{Chapter 4: Exercise 5d}
\begin{lstlisting}
library('tseries'); library('urca'); library('vars')
rAll=read.csv("~/Documents/Personal/Stats242/Hw/Hw2/Rates.csv")
sAll=read.csv("~/Documents/Personal/Stats242/Hw/Hw2/d_SP500.csv")
excessReturn=0.03/252;
usd1<- rAll$USD_by_INR
pound1<-rAll$GBP_by_INR
euro1<-rAll$EUR_by_INR
x1=log(usd1); x2=diff(x1); usd=c(0,x2)
x1=log(pound1); x2=diff(x1); pound=c(0,x2)
x1=log(euro1); x2=diff(x1); euro=c(0,x2)
rAll$usd=usd;
rAll$pound =pound;
rAll$euro=euro;
#Find all the elements that are common
s1=intersect(sAll$caldt, rAll$Date)
r=subset(rAll, rAll$Date %in% s1)
s=subset(sAll, sAll$caldt %in% s1)
length<- nrow(r)
usd=r$usd
pound=r$pound
euro=r$euro
market=s$vwretd
usdInitial=r$USD_by_INR
poundInitial=r$GBP_by_INR
euroInitial=r$EUR_by_INR
#This function returns end of month indexes of end of month
getEndOfMonthIndices=function(dateSeries){
l=length(dateSeries)
dateT=as.character(dateSeries)
d=as.Date(dateT,"%Y%m%d")
m=months(d)
endOfMonth=rep(-1,l)
curMonth=1;
for(i in 2:l)
{
if(m[i]!=m[i-1])
{endOfMonth[i-1]=curMonth; curMonth=curMonth+1}
}
return((list(months=m, endOfMonth=endOfMonth)))
}
#
e= getEndOfMonthIndices(r$Date)
endOfMonthIndex= e$endOfMonth
getLast3MonthsOfdata=function(curMonth){
curIndex= which(endOfMonthIndex==curMonth)
startMonth=curMonth-3
startIndex= which(endOfMonthIndex==startMonth)+1
if(startMonth==0){startIndex=1}
usd.3mths<-usd[startIndex:curIndex]
pound.3mths<-pound[startIndex:curIndex]
euro.3mths<-euro[startIndex:curIndex]
market.3mths<-market[startIndex:curIndex]
usd.orig<-usdInitial[startIndex:curIndex]
pound.orig<-poundInitial[startIndex:curIndex]
euro.orig<-euroInitial[startIndex:curIndex]
return(list(usd.3mths= usd.3mths, pound.3mths= pound.3mths, euro.3mths= euro.3mths,market.3mths=market.3mths,usd.orig= usd.orig, }
getNext3MonthsOfdata =function(curMonth,u){
curIndex= which(endOfMonthIndex==curMonth)
endMonth=curMonth+ eligibilityPeriod
if(endMonth>47){endMonth =47}
endIndex= which(endOfMonthIndex== endMonth)
usd.3mths<-usd[curIndex:endIndex]
pound.3mths<-pound[curIndex:endIndex]
euro.3mths<-euro[curIndex:endIndex]
mkt.3mths<-market[curIndex:endIndex]
usd.orig<-usdInitial[curIndex:endIndex]
pound.orig<-poundInitial[curIndex:endIndex]
euro.orig<-euroInitial[curIndex:endIndex]
return(list(usd.3mths= usd.3mths, pound.3mths= pound.3mths, euro.3mths= euro.3mths,usd.orig= usd.orig, pound.orig= pound.orig, }
getAPT=function(s1,s2,mkt,s1.orig, s2.orig){
s1.ex=s1-excessReturn
s2.ex=s2-excessReturn
mkt.ex=mkt-excessReturn
m1<-lm(s1.ex ~ mkt.ex)
alpha1<- coef(m1)[1]
beta1 <- coef(m1)[2]
m2<-lm(s2.ex ~ mkt.ex)
alpha2<- coef(m2)[1]
beta2 <- coef(m2)[2]
dist =(alpha1-alpha2)^2+ (beta1-beta2)^2
#Let's keep the difference between the two series-squared as our delta
stdPD=sd((s1.ex-s2.ex)^2)
x1= s1.orig/s1.orig[1]
x2= s2.orig/s2.orig[1]
stdPD=sd((x1-x2)^2)
return(list(pd= dist, stdPD = stdPD))
}
#StartingBalance
StartingBalance=1000;
KTradingBasis=0;
Amount=100
runSimulation=function(name, series1, series2, delta, series1.Orig, series2.Orig)
{
trade=list();
l=length(series1)
trade$name=name;
trade$delta=delta;
trade$series1=series1-excessReturn;
trade$series2=series2-excessReturn;
#This is series1 shares
series1Shares= rep(0,l);
series2Shares= rep(0,l);
cash=rep(StartingBalance,l);
netLiquidAssets=rep(StartingBalance,l);
pnl=rep(0,l);
dailyReturn=rep(0,l)
series1= series1.Orig/series1.Orig[1]
series2= series2.Orig/series2.Orig[1]
############
#
# Now we run simulation...
# -Series were co-integrated, so we run simulation here...
#
##############
amount= Amount
tradeTime=0
biggerSeries=0;
maxTime=l;
#This is eligibilityPeriod + Convergence Period
if(eligibilityPeriod>1)
{maxTime=l-21;} #Reserve last 21 days for convergence period only
for(i in 2:l)
{
series1Shares[i]= series1Shares[i-1]
series2Shares[i]= series2Shares[i-1]
theDifferenceSquared=(series1[i]-series2[i])^2
if (tradeTime==0 && (theDifferenceSquared>2*delta))
{
tradeTime=i; #We are now in trade
print(sprintf("TradeStart=%f",tradeTime))
proportion= amount/(series1[i]);
if(series1[i]<series2[i])
{
series1Shares[i]= amount/series1.Orig[i]; #Use non-normalized
series2Shares[i]= -amount/series2.Orig[i]; #Use non-normalized
biggerSeries =2;
}
if(series1[i]>series2[i])
{
series1Shares[i]= -amount/series1.Orig[i]; #Use non-normalized
series2Shares[i]= amount/series2.Orig[i]; #Use non-normalized
biggerSeries=1;
}
}
#Check if pair has converged...
if(tradeTime >0)
{
resetVariables=0
if(biggerSeries==1 && (series1[i]<series2[i]))
{resetVariables=1;}
if(biggerSeries==2 && (series1[i]>series2[i]))
{resetVariables=2;}
#Reset all variables
if(resetVariables>0)
{
print(sprintf("TradeEnd=%f,Reason=%f, pnl=%0.4f",i, resetVariables,pnl[i-1]))
series1Shares[i]=0;
series2Shares[i]=0;
biggerSeries=0;
tradeTime=0
}
}
if(i==l)
{
print(sprintf("TradeEnd=%f,Reason=NoConvergence, pnl=%0.4f",i, pnl[i-1]))
series1Shares[i]=0;
series2Shares[i]=0
biggerSeries=0;
tradeTime=0
}
#Calulate the transaction Cost...
tv1= series1.Orig[i]*(series1Shares[i]-series1Shares[i-1])
tv2= series2.Orig[i]*(series2Shares[i]-series2Shares[i-1])
tc= (abs(tv1)+abs(tv2))* KTradingBasis
cash[i]=cash[i-1]-series1.Orig[i]*(series1Shares[i]-series1Shares[i-1])-series2.Orig[i]*(series2Shares[i]-series2Shares[#Calculate PNL in hand
netLiquidAssets[i]=cash[i] + series1.Orig[i]* series1Shares[i] + series2.Orig[i]* series2Shares[i]
dailyReturn[i]=(netLiquidAssets[i]-netLiquidAssets[i-1])/netLiquidAssets[i-1]
pnl[i]= netLiquidAssets[i]-StartingBalance;
}#end of simulation...
trade$series1Shares= series1Shares
trade$series2Shares= series2Shares
trade$netLiquidAssets= netLiquidAssets
trade$dailyReturn = dailyReturn
trade$cash=cash
trade$pnl=pnl
return(trade)
}
#The Maximum simlutanoes trade possible at a time =4
runTrade=function(u,v)
{
usdAndPound= getAPT(u$usd.3mths, u$pound.3mths, u$market.3mths,u$usd.orig,u$pound.orig)
usdAndEuro= getAPT(u$usd.3mths, u$euro.3mths, u$market.3mths,u$usd.orig,u$euro.orig)
poundAndEuro= getAPT(u$pound.3mths, u$euro.3mths, u$market.3mths,u$usd.orig,u$euro.orig)
minDist=min(usdAndPound$pd, usdAndEuro$pd, poundAndEuro$pd)
minTrade=1
if(usdAndEuro$pd ==minDist){minTrade = 2}
if(poundAndEuro$pd ==minDist){minTrade=3}
if(minTrade==1)
{
print ("USD And Pound")
trade= runSimulation(name="usaAndPound",v$usd.3mths,v$pound.3mths, usdAndPound$stdPD, v$usd.orig, v$pound.orig)
}
if(minTrade==2)
{
print ("USD And Euro")
trade= runSimulation(name="usdAndEuro",v$usd.3mths,v$euro.3mths, usdAndEuro$stdPD, v$usd.orig, v$euro.orig)
}
if(minTrade==3)
{
print ("Pound And Euro")
trade= runSimulation(name="poundAndEuro",v$pound.3mths,v$euro.3mths, poundAndEuro$stdPD, v$pound.orig, v$euro.orig)
}
return(trade)
}
#############################################################################
#
#
#############################################################################
# Now, let's start trading
eligibilityPeriod=3
KTradingBasis=0;
getResults=function()
{
# Now let's start pairs trading for each month...
totalNumberOfMonths=max(endOfMonthIndex)
monthsAvailableForTrading= totalNumberOfMonths-eligibilityPeriod
profit=0; pr=0;
curMonth=3;
PNL=0;
PortfolioReturns=0;
while(curMonth <= monthsAvailableForTrading){
print(sprintf(">CurMonth=%0.0f, pr=%0.4f, profit=%0.4f",curMonth,pr, profit))
u=getLast3MonthsOfdata(curMonth)
v=getNext3MonthsOfdata(curMonth,u)
trade=runTrade(u,v)
lastPNL=PNL[length(PNL)];
thisPNL=trade$pnl
overallPNL= lastPNL+ thisPNL
PNL=c(PNL, overallPNL)
PortfolioReturns=c(PortfolioReturns,trade$dailyReturn)
pr=trade$pnl[length(trade$pnl)]
profit=profit+pr
print(sprintf("<CurMonth=%0.0f, pr=%0.4f, profit=%0.4f",curMonth,pr, profit))
curMonth=curMonth+eligibilityPeriod
}
sharpeRatio=(mean(PortfolioReturns)-0.03/252)/sd(PortfolioReturns)
sharpeRatio.Ananualized= sharpeRatio*sqrt(252)
print(sprintf("Profit=%0.4f, SharpeRatio=%0.4f, sharpeRatio.Ananualized=%0.4f", profit, sharpeRatio, sharpeRatio.Ananualized))
}
# Now, let's start trading
eligibilityPeriod=1; KTradingBasis=0;
getResults();
eligibilityPeriod=3; KTradingBasis=0;
getResults();
eligibilityPeriod=1; KTradingBasis=0.0015;
getResults();
eligibilityPeriod=3; KTradingBasis=0.0015;
getResults();
\end{lstlisting}
\textbf{R-Program for Bollinger Bands:}
\begin{lstlisting}
rd= read.csv("royaldutch.csv",header=T)
colnames(rd) = c("Date","Close")

## compute Bollinger Bands on price and scaled volume
computeBollingerBands <- function(dat, ndays=20, nsd=2)
{

  ## create a (normalised, but that's just candy)
      weight vector
  weights <- rep(1/ndays,ndays)
  weights2<- rev(1:ndays)
  weights2<-weights2/sum(weights2)
  ## and apply it as a one-sided moving average
     calculations, see help(filter)
  bbmiddle <- as.vector(filter(dat$Close,
            weights, method="convolution",side=1))
  bbmiddle2<- as.vector(filter(dat$Close,
   weights2, method="convolution",side=1))
  ## use var(x) = E(x^2) - E(x)^2 to compute
     rolling variances
  v <- filter(dat$Close^2, weights, method="convolution",
              side=1) - bbmiddle^2
  ## from which we calculate rolling standard deviations
     the usual way
  bbsd <- as.vector(sqrt(v))

  bbupper <- bbmiddle2 + nsd*bbsd   # upper Bollinger band
  bblower <- bbmiddle2 - nsd*bbsd   # lowet Bollinger band
  ## now extend the data frame with a few new columns
  D <- cbind(dat, bbmiddle, bbmiddle2,bbsd, bbupper, bblower)

  D                        # return the augmented data frame
}

bbresult<- computeBollingerBands(rd, ndays=20, nsd=2)

#using band to build trading signals
buysignal <- rd$Close< bbresult$bblower
sellsignal <- rd$Close> bbresult$bbupper
tradeindicator <- rep(0,dim(rd)[1])
tradeindicator[buysignal]=1
tradeindicator[sellsignal]=-1

signal2logRet<- function (Close, signal){
	longposition = which(signal==1)
	shortposition = which(signal==-1)
	if(length(longposition)==0) return(NA)
	lastTrade = longposition[1]
	i=longposition[1]+1
\newpage
	position = 1 #initial long position
	value = 1 #initial total value is 1
	#find te first long position
     while(i<=length(Close)){
     	if(signal[i]==position*(-1)){
     		# set new last trading
     		position = signal[i]#change position
     		if(signal[i]==-1){#liquidate
     		value = value * (Close[i]/Close[lastTrade])
     		#print(c(i,Close[i],Close[lastTrade],value))
     			}  	
     		lastTrade = i 	
     		}
     		i = i+1
     	}
     log(value)
	}
signal2logRet(rd$Close,tradeindicator)	
 #gives a log return of 0.96
\end{lstlisting} 
\end{small}