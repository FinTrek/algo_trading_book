% !TEX root = ../../book.tex
\chapter{Introduction}\label{chap:ch_intro}
\section{About This Book}

Several books, in recent years, have been published  on the topics of Algorithmic Trading and Quantitative Strategies. Some books provide broad non-quantitative overviews of the subject while others, a strictly academic, quantitative, treatment of one or more advanced and theoretical topics. While there is a lot of excellent information in these books, it is the opinion of the authors that none achieve the goal of providing the reader with the full picture, from the business understanding point of view, the real, practical, challenges faced by a quantitative researcher (aka Quant) in this space. We believe there are not many publications focused on providing both the hands-on and quantitative toolkit necessary to effectively operate in this field. This book strives to achieve exactly this goal. It is written by one academic that has studied and taught the methods necessary for the student of quantitative finance for over 20 years and two quantitative practitioners with a combined 25+ years of experience in Electronic Trading who have worked (and still work) for some of the largest trading operations in the world. Bringing together these two unique perspectives give the authors the opportunity to provide a much more practical and integrated view of the field.


The book will provide an inside look into the world of Electronic Trading, what are the actual challenges, most often ignored by more academic treatment of the subject, and how practitioners have solved these problems in the past and new ideas on how they plan to solve them in the future. The subject is then significantly deepened by a much more formal mathematical treatment in order to provide the necessary quantitative methods as well as a targeted review of the relevant academic literature to guide the student keen to deepen her knowledge.


By construction (as well as by the obvious writing limitations of the authors) the book will thus present two very distinctive writing styles: one more high level, jargon ridden and conspicuously non rigorous, the other, a lot more formal and mathematical. This is not only the result of the different background and focus of the authors but is also a natural reflection of the dynamics typical of quantitative researcher on a trading floor where expediency, time to market, and other practical considerations often trump rigor and aesthetics of highly mathematical models for intuitively simpler approach. We hope that the reader embraces this dual writing style as their own.


Even on the more quantitative subjects this book is somewhat different. In the spirit of providing the reader with the hands-on tools, this book will be accompanied by a collection of practical Jupyter Notebooks where some of these methods are tested against real data. This will allow the reader to go beyond the theory into the actual implementation and guide how to leverage existing libraries. Wherever possible the charts and tables in the book are generated directly by these notebooks and the data-sets provided bring further life into the treatment of the subject. In the excellent spirit largely embraced by the academic/quantitative literature, the notebooks as well as the data and the latex document of this book are made  available for free on: \url{https://github.com/NehrenD/algo_trading_book}.


Working through this book the reader should, in principle, be able to create all the building blocks needed for a full solution. She will also understand the context and challenges she will face in real life and why things are the way they are. We hope that the reader will enjoy this journey as much as the authors have enjoyed creating this ``guided visit'' into the workings of Electronic Trading and Quantitative Strategies, one of the most active and exciting spaces in the world of finance.



\section{How The Book is Structured}

This book is divided into five parts. 

\begin{itemize}
\item Part 0 will set the stage for our journey. After a whirlwind tour through the history and evolution of Equity Trading we will delve in a review of the current features of modern Market Structure in all its glory, chaos and (over-)complexity. This gives the reader context on the business aspects of trading in order for her to understand why things work the way they do. The next section will provide a brief high level overview of market micro-structure which tries to explain and model the dynamics of a trading venue which is heavily influenced by the principal technical mechanism in which trading is taking place: The time-price priority limit order book with  continuous double auction. This will set the stage for the introduction of a critical but elusive concept in trading: Liquidity and the role of market makers. Part 0 will conclude with high level concepts in Algorithmic Trading.

\item Part 1 will provide the foundational theory of stochastic processes and it's relation to time series models as it applies to equity trading and one of the central datasets in finance: market data. We will have look at both univariate and multivariate time series models of both mean and variance of asset returns. While somewhat less used today these models are important foundationally as conceptual frameworks and act as baselines for more advanced methods. The last chapter of Part 1 will look at some more advanced topics like State-Space Models and application of some of Machine Learning methods.

\item In Part 2, we dive into the the broad topic of Quantitative Trading. The aim of this part is to introduce the reader to the toolkit necessary to get started in this field. From some historical perspective to general approaches to Alpha generation to critical insights into the art of backtesting and it's mine fields are covered. Since most quantitative strategies are portfolio based, meaning that alphas are usually  combined  and optimized over a basket of securities, we will introduce the topic of Active Portfolio Management and Mean-Variance optimization and more advanced topics of Dynamic Portfolio Selection. We end this part discussing a topic that, while somewhat orthogonal to the subject, is concerned in one of the newer alternative datasets: News and Sentiment analytics.

\item Part 3 is dedicated to Execution Algorithms. In a sense Execution Algorithms are a subset of Quantitative Trading Strategies but have evolved separately to go from a set of mechanical workflow tools to a multibillion dollar business that today trades with little or no human interaction, the rebalancing flow of thousands of Investment Companies and Hedge Funds that combined manage many trillions of dollar in securities. This section provides the a review of the actual mechanics of trading. 

\item Finally, Part 4 deals with some other aspects of developing both quantitative trading strategies and execution algorithms. Trading is a highly technological process that requires the integration of multiple technologies and systems: from market data feeds, to exchange connectivity, to low-latency networking and co-location, to back-office booking and reporting. Developing a modern and performing trading platform requires a lot of considerations and trade-offs. Additionally automated trading in particular is a highly regulated activity with increasing oversight by many regulatory bodies. The role of Compliance, Market Risk and, in recent years, Model Risk and Operational Risk organizations have become more and more central to this process which requires the practitioners to navigate a  complex web of requirements, policies and social interactions.
\end{itemize}


