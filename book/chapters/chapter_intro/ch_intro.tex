% !TEX root = ../../book.tex
\chapter{Introduction}\label{chap:ch_intro}
\section{About This Book}

In recent years several books have been published  on the topics of Algorithmic Trading and Quantitative Strategies. Some books provide broad non-quantitative overviews of the subject while others are strictly academic, treatments of theoretical topics. These books hold a wealth of  information. We hope to in this book, expose the reader to the more complete picture, including the business aspects, the practical challenges, and the essential tools needed for a quantitative researcher (aka Quant) to effectively operate. It is jointly written by an academic and two senior quantitative practitioners with experience in Electronic Trading. Bringing together these diverse perspectives, we provide a unique, practical, and integrated view of the field.


This book gives an inside look into the current world of Electronic Trading and Quantitative Strategies. We address actual challenges by presenting intuitive and innovative ideas on how to approach them in the future. The subject is then strengthened through a more formal treatment of the necessary quantitative methods with a targeted review of relevant academic literature. This dual approach is also reflective of the dynamics, typical of quants working on a trading floor where commercial needs, such as time to market, often supersede consideration of rigorous models in favor of intuitively simple approaches. 


Our unique approach in this book is to provide the reader with the hands-on tools. This book will be accompanied by a collection of practical Jupyter Notebooks where select methods are applied to real data. This will allow the reader to go beyond theory into the actual implementation, while familiarizing them with the libraries available. Wherever possible the charts and tables in the book are generated directly from these notebooks and the data-sets provided bring further life to the treatment of the subject. We also add exercises that the students can work through on their own. The notebooks as well as the data and the exercises are made  available for free on: \url{https://github.com/NehrenD/algo_trading_book}. While reading and working through this book, the reader should be able to gain insight into how the field of Electronic Trading and Quantitative Strategies, one of the most active and exciting spaces in the world of finance, has evolved.



% How The Book is Structured
\section{How The Book is Structured}

This book is divided into five parts. 

\begin{itemize}
\item Part~0 sets the stage. We narrate the history and evolution of Equity Trading and delve into a review of the current features of modern Market Structure. This gives the reader context on the business aspects of trading in order for them to understand why things work as they do. The next section will provide a brief high-level foundational overview of market microstructure which explains and models the dynamics of a trading venue heavily influenced by the core mechanism of how trading takes place: the price-time priority limit-order book with continuous double auction. This will set the stage for the introduction of a critical but elusive concept in trading: Liquidity.

\item Part~I provides an overview of time series models applied to equity trading. We will address univariate and multivariate time series models of both mean and variance of asset returns, and other associated quantities such as volume. While somewhat less used today because of high frequency trading, these models are important  as conceptual frameworks, and act as baselines for more advanced methods. The last chapter of Part~I will present more advanced topics like State-Space Models and modern Machine Learning methods.

\item Part~II dives into the broad topic of Quantitative Trading. Here we provide the reader with a toolkit to confidently approach the subject. Historical perspectives from Alpha generation to the art of backtesting are covered here. Since most quantitative strategies are portfolio based, meaning that alphas are usually combined  and optimized over a basket of securities, we will briefly introduce the topic of Active Portfolio Management and Mean-Variance Optimization, and the more advanced topic of Dynamic Portfolio Selection. We conclude this section discussing a somewhat recent topic:  News and Sentiment Analytics. Our intent is to also remind the reader that the field is never ``complete'' as new approaches (such as this from behavioral finance) are embraced by practitioners once the data is available.

\item Part~III covers Execution Algorithms, a sub-field of Quantitative Trading which has evolved separately from simple mechanical workflow tools, into a multi-billion dollar business. We begin by reviewing various approaches to modeling trade data, and then dive into the fundamental subject of Market Impact, a complex and least understood concept in finance. Having set the stage, the final section presents a review of the evolution and the current state of the art in Execution Algorithms.

\item Finally, Part~IV deals with some technical aspects of developing both quantitative trading strategies and execution algorithms. Trading has become a highly technological process that requires the integration of numerous technologies and systems,  ranging from market data feeds, to exchange connectivity, to low-latency networking and co-location, to back-office booking and reporting. Developing a modern and high performing trading platform requires thoughtful consideration and some compromise. In this part, we look at some important details in creating a full end-to-end technology stack for electronic trading. We also want to emphasize the critical but often ignored aspect of a successful trading business: Research Environment.
\end{itemize} 

%%%%%%%%%%%%%%%%
% To be removed
\newpage

\todo{Target audience: More than a pure text book, we’re targeting the book to be a reference for both graduate students studying Quant finance and practitioners in the field.}

\todo{The book needs a thorough scrubbing of all the content. Tightening all the language and making sure things are flowing}

\todo{Chapter 2: - enhance the academic treatment of microstructure (rigorously pointing to more recent papers - Need to make sure we ensure that we do not discuss items we have not previously presented. For example I mentioned the SEC in Chapter 2 but never discussed it. Also there is a statement about price discovery where secondary markets are discuss but never defined. We need to assume that the reader are not familiar with any of this}

\todo{Chapter 3/4: - shrink the “depth” of the time series chapter (e.g mathematical derivations, etc) to make it easier to read.  - instead of derivations, focus more on hands on examples that readers can play with in python. Chapter 3 is done; Daniel will focus on examples.}
 
\todo{Chapter 5: as part of advanced topic, create a section on the practicalities of adapting standard daily time series models to the intraday space: normalizing factors, change point detection, stock specific characteristic time, etc.)}
 
\todo{Chapter 8: content is good, but we should merge it into another chapter. Chap 9 looks like the best candidate, where we can make News Analytics a deeper section as one particular type of dataset/Analytics. (Raja would like to keep Chapter 8 as it stands.)}
 
\todo{Chapter 12/13: -to be cleaned up and combined I to one single chapter - in the new chapter, add a section on Regulatory aspects}
 
\todo{To be cleaned up: - MI section: include some of the latest content like Bouchaud 2019 on multi day decay - Portfolio section: quite good already, just need some clean up, and maybe this is an other place where to discuss taking a standard approach (daily time series for risk models like Barra) and making them more dynamic intraday.}

\todo{General: Keep the consistency on the content. Where possible have the more practitioner perspective up-front and add more quantitative treatment afterwards. I think Chapter 7 could use a bit of that since it was almost all written by Raja}
%%%%%%%%%%%%%%%%