% !TEX root = ../../book.tex
%\hfill
%\par\vspace{\baselineskip}

\chapter{The Technology Stack}
There are a broad spectrum of trading infrastructures and operations but none, we think, is as broad and complex as the one in a large electronic trading execution operation in a large Broker-Dealer. So we use that as a template for our exploration of the technology stack that is necessary to support such an operation. We start by reviewing the end to end flow of information and then go into some detail on the various components. As previously discussed all large ET businesses also operate their own ATS so we'll briefly look at the technology setup needed to support that use case as well.
\section{From client Instruction to Trade Reconciliation}
TODO Build an end-to-end diagram

Figure~\ref{fig:pexchrate} shows the end to end diagram of a hypothetical trading infrastructure. We can use this diagram to follow the full lifecycle of a client order and the main components of the infrastructure involved.
\subsection{Client Side}
The trader enters an algorithmic order into her Execution Management System (EMS) a specialized trading infrastructure that integrates with internal systems and provides all the tools an institutional traders needed to manage her day to day trading needs. EMS are most often vendor products with Ez Castle \footnote{\url{https://www.ezesoft.com/}}, Portware \footnote{\url{http://www.portware.com/}}and FlexTrade \footnote{\url{https://flextrade.com/}} some of the most popular options. Many larger buy-side trade organizations have more complex needs and often have an homegrown solution.\\

The trader chooses from a drop-down menu the particular provider and Algo strategy and  chooses the desired parameters. Every broker that wants to expose their execution into an EMS needs to certify with the EMS by providing a FIX Specification Document that highlights what strategy are available and what parameters are available for each and the validation information about the parameter. The EMS integrates this into their system and exposes them in the front end for the trader to chose. FIX stands for \emph{Financial Information eXchange} and it's a standard communication protocol specifically designed for financial applications \footnote{\url{https://www.fixtrading.org/}}\\

Once the order is submitted the EMS uses a preconfigured network connection to transfer the order to the broker. This session is created as part of the \emph{Client Onboarding} step a quite laborious process to setup a new client relationship, configure the client financial limits and controls.

\subsection{Inbound Gateway}
We are now broker-side where the order is received by and Inbound gateway. The first think to happen is some form of validation to ensure that the message is a valid order and has all the necessary fields. Next step is to perform a set of risk and credit checks to ensure that the order is within the specified risk limits and the client overall exposure, the maximum notional the client is allowed to trade and have in the market at any point in time. Any mismatch in these validation steps and the order is rejected back to the client.

\subsection{Order Management System and Order Enrichment}
Once the order is validated it is created within the broker dealer infrastructure. Order life-cycle is quite complicated and nuanced and it is critical that the state of the order is always up to date and state transitions carefully managed. The role of creating and managing the state of the order is fulfilled by an Order Management System (OMS). These infrastructures are at the heart of any trading operation and are one of the most important components. They are often built around a Bus based architecture a software paradigm where loosely couple components communicated with each other over a messaging middleware. The advantage of this approach is that other ancillary components within the infrastructure can ``Listen in" on the messages and use the information to affect other systems or collect the data for trade reporting, and analysis.\\

One step often performed in this layer is Order Enrichment. This is a step that adds additional, lower level parameters that adjust and customize the execution behavior for the specific client/algorithm/parameters triplet. This step also translates nuances between the instruction the client sends and what the execution system understands. This is often accomplished by a specialized \emph{Rules Engine} a software library that allows a set of rules to be applied to an order and then re-evaluate the rules after any modification until no rules are triggered.

\subsection{Execution Strategy Stack}
And this is, as the famous expression quote: "Is where the magic happens!" The order reaches the software component where the Algo is actually executed. We'll delve into more details in on the execution stack in the next section so here we just assume that the strategy is initialized and starts executing. The strategy is associated with the order contained in the OMS often called the ``Parent Order" and as consequence of the strategy logic one or more ``child" orders  are created in the OMS and the sent forward for submission to one or more trading venue. Before the order are actually forwarded the order passes through an additional control layer to ensure the strategy does not violate the risk limits and speed bumps, general term used for limits that prevents the strategy from trading too fast or too aggressively or send too many child orders, etc.

\subsection{Outbound Gateway}
The child orders are received by another piece of infrastructure that is responsible for actually sending the orders to the market: The Outbound Gateway. All venues support one more protocols to communicate with market participants. Essentially all of them support a FIX protocol but in most cases they also support a much faster ``native" protocol that encodes the instructions in a compressed binary protocol. The role of the Outbound Gateway is to connect to the various venues and then act as a translation layer from the internal representation in the OMS to the external representation of the specific protocol implemented by the venue. The outbound gateway also listens to the connection callback to capture any asynchronous event coming from the venue like order insert/cancellation acknowledgements and executions events, and updates the state of the child order representation in the OMS.

\subsection{Notifications to the Client}
As the strategy executes orders in the market the OMS keeps the state of the parent order up-to-date and either every execution or periodically send updates back via the inbound gateway back to the client's EMS that update its own state and provide feedback to the trader that the strategy is executing,  what the average prices achieved, and other analytics necessary for the trader to understand how well the strategy is executing. The inbound control layer is also kept uptodate so that the total state of all client orders is accounted for if/when a new order from the same client is received.\\

\subsection{Back Office}
We are almost done. The step above completes the realtime feedback loop from the client through the executing strategy to the market and back. The rest of the processing is in most cases done offline by a set of infrastructures commonly referred as ``Back Office Systems". These systems play a critical role in the business and regulatory side of trading:

TODO Get clear picture of these steps.

\section{Algorithmic Trading Infrastructure}
Algorithms are in most cases built on top of a software framework called ``Strategy Container", a piece of infrastructure that receives a new order instruction from the OMS and instantiates the code that embeds the particular strategy, initializes the specific parameters and the kicks of the execution. These strategy container provides to the s
\subsection{Market Data Plant}
\subsection{Execution Strategy and Order  Routing Container}
\subsection{Control Layer}
\subsection{Additional Infrastructures}
\subsubsection{Realtime Analytics Engine}
\subsubsection{Algorithm Switching Engine}
\subsubsection{Portfolio Algorithm Engine}

\section{Other Algorithmic Trading Usecases}


\section{ATS Infrastructure}
\subsection{Matching Engine}
\subsection{Client Tiering and other Rules}

