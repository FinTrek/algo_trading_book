% !TEX root = ../../exercises.tex
\chapter{Statistical Trading Strategies and Back-Testing \label{ch:stat_ts}}

%Problem
\prob {\it Technical Rules} Consider the data on exchange rates (in file \texttt{rates.csv}) between the Indian Rupee and the US Dollar, British Pound and Euro; it is daily data from December 4, 2006 to November 5, 2010. In order to compare the performance of a strategy, appropriately normalize the entry point exchange rate: invest equal amounts of capital (counted in INR, which we will call the ``base currency'') in each trade. Total portfolio return on each day should be computed as a weighted sum of returns from each active trade (weighted by the current value of each trade). Evaluate the signals discussed in (a) and (b) below using the Sharpe ratios and PNL with and without including transaction costs of $15$ basis points. Assume a risk free rate of $3\%$. (1 basis point = $10^{-4}$; so each dollar traded costs $0.0015$ in transaction costs)
	\begin{enumerate}[(i)]
	\item{\it Moving average trading rules}: Trading signals are generated from the relative levels of the price series and a moving average of past prices.
		\begin{itemize}
		\item  Rule 1: Let $\overline{p}_t^{(m)}$ as defined in \eqref{eqn:multi}; use the strategy to sell if $p_t>\overline{p}_t^{(m)}$ and to buy if  $p_t<\overline{p}_t^{(m)}$. Evaluate this strategy. What is the optimal `$m$'?
		\item Rule 2 (Bollinger rule): Use the Bollinger rule as outlined in the text; what is the optimal `$m$'?
		\item Rule 3 (Resistance-Support): Buy if $p_t > \max_{\substack{0\le i< m}}p_{t-i}$ and sell if $p_t < \min_{\substack{0\le i< m}}p_{t-i}$. Evaluate this rule. What is the optimal `$m$'?
		\item Rule 4 (Momentum): Compute a short-term moving average over the last $m$ prices, $\overline{p}_t^{(m)}$ and a long-term average over the last $n$ prices, $\overline{p}_t^{(n)}$ with $m<n$. If $\overline{p}_t^{(m)}$ crosses $\overline{p}_t^{(n)}$ from below, sell and if it crosses $\overline{p}_t^{(n)}$ from above, buy. Evaluate this strategy. What are the optimal values of $m$ and $n$?
		\end{itemize}
	\item{\it RSI Oscillator Rule}: Use $\text{RSI}_t$ as defined in \eqref{eqn:rsi}. \hfill \break
If $\text{RSI}_t<30$, buy and if $\text{RSI}_t>70$, sell. Evaluate this trading strategy. What is the optimal `$m$'?

	\item {\it Pairs Trading: Distance Method}

We want to develop a pairs trading strategy and check if it does any better than  the strategies in Problem 1. At the last trading day of the month, look back 3 months and identify if any pairs are worth trading. Use the distance measures, thresholds, as given in the text.

	\item {\it Pairs Trading: Cointegration Method}

In the cointegration method, we can consider the possibility of trading more than 2 assets. Check using a moving window of 3 months which series are cointegrated. Develop appropriate trading strategies to trade portfolios formed from the cointegrating vectors and compare with the pairs trading strategies from based on other criteria. Concretely define the trading indicators and thresholds for entry and exit. Summarize the performance results.

	\item {\it Pairs Trading: APT method}

Consider the data with the S\&P500 market index returns and assume that the risk free rate is 3\%. Regress each currency's excess return (over the risk free rate) on the excess market return over a moving window of 3 months. Find pairs based on the similarity of the regression coefficients.
	\begin{enumerate}
	\item Compare the pairs that result from the APT method with those obtained in Problem 2.
	\item After finding the pairs, use the same rules as in Problem 2 to decide when to enter and exit specific trades.
	\end{enumerate}
Summarize the performance results.

	\item {\it Summing up}

Provide a summary table of portfolio returns of all methods with the main findings and drawbacks of all the methods that are explored in this assignment. In particular, comment on the performance of single-asset strategies versus multi-asset strategies. Also comment on whether the market factor adds any value.
	\end{enumerate}


%Problem
\prob The file Exchange rates.csv contains exchange rates between US\$ and 25 major currencies; the daily data spans from Jan 3, 2000 to April 10, 2012. Consider the data from Jan 3, 2002 for the following currencies: EUR, GBP, DKK, JPY and CNY. Compute $r_t = \ln{p_t} - \ln{p_{t-1}}$, returns for each series. \\


\indent\textbf{A. Evaluation of Trading Rule:} Consider the data starting June 11, 2003.
	\begin{enumerate}[(a)]
	\item Compute the average return for each day. We will treat this a `market' return.
	\item Consider the two currencies: EURO and JPY and assume that your investment is restricted to only these two currencies. We will follow the simple algorithm: if 
	\[
	\frac{r_i - \overline{r}}{sd(r_i)}=
	\begin{cases}
	>2, & \text{sell} \\
	< -2, & \text{buy} \\
	\text{otherwise}, & \text{hold}
	\end{cases}
	\]
The average and standard deviation of the returns are based on last 30-day observations. Evaluate this algorithm by comparing it with simple holding.
	\item What is the value of the `market' return? How could this information be used?
\end{enumerate}


\textbf{B. Trading Rule with other factors:} In addition to exchange rates from June 11, 2003, consider the data on commodity prices: Gold, Oil, Copper and a commodity index. Among the following currencies, CAD, BRL, ZAR, AUD, NZD, CAF and RUB, which are most influenced by the commodity prices. Discuss how you would incorporate this into a trading strategy. \\


%Problem 
\prob This is an exercise to replicate the pairs trading strategy. Obtain the daily data from January 2000 to June 2014 on stock prices, returns, trading volume and shares outstanding on the following companies only: \\

\indent Oil stocks: HAL, SLB, EQT, OXY, CVX \\

\indent Tech stocks: AAPL, CSCO, GOOG, INTL, MSFT \\

\indent Bank stocks: BAC, JPM, WFC, C, STI \\

\indent Retail stocks: ANF, GES, SKS, SHLD, URBN \\

Begin forming pairs in January 2000 based on the estimation period, January 2000 to December 2000. The last estimation period is January 2014 to December 2014. Eligible pairs corresponding to the last estimation remain eligible till December 2014. Pairs that opened in December 2013 and did not converge with the stipulated maximum of six months would have been closed in June 2014.


As you recall, the pairs trading strategy works as follows:
	\begin{enumerate}[--]
	\item Within each sector, match pairs based on normalized price differences over the past year.
	\item When a tracked pair diverges by more than two standard deviations of their normalized price difference, buy the ``cheap'' stock and sell the ``expensive'' one after a one-day wait period (to mitigate the effects of any market irregularities).
	\item If prices for a traded pair re-converge, exit the paired positions.
	\item If prices for a traded pair do not re-converge within six months, close the positions. An alternative strategy closes all traded pairs in ten trading days.
	\end{enumerate}

\noindent Questions of interest:
	\begin{enumerate}[(a)]
	\item Quantify the profitability of the pairs trading strategy and on average, how long it takes for pairs to re-converge?
	\item Can you provide some insight into why the strategy works based on trading volume, shares outstanding and the market trend?


Another Question of interest
	\item Is there a pair across sectors that can outperform the pairs formed within each sector? \\
	\end{enumerate}


%Problem
\prob  This is another exercise on `pairs trading' using CRSP data. Obtain the daily data from January 1992 to June 2010 on stock prices, returns, trading volume and shares outstanding on the following companies only:

\indent DELL, HP, ADBE, MSFT, AAPL\hfill \break
\indent JPM, WFC, BAC, C, BRK.A\hfill \break
\indent JNJ, WMT, COST, KO, PEP\hfill \break
\indent GM, TM, F\hfill \break
\indent XOM, COP 

``Begin forming pairs in January 1993 based on the estimation period, January 1992 to December 1992. The last estimation period is January 2009 to December 2009. Eligible pairs corresponding to the last estimation remain eligible till December 2009. Pairs that opened in December 2009 and did not converge with the stipulated maximum of six months would have been closed in June 2010.'' \\

\noindent Questions of interest:
	\begin{enumerate}[(a)]
	\item Quantify the profitability of the pairs trading strategy and on average, how long it takes for two pairs to reconverge?
	\item Can you provide some insights into why the strategy works based on trading volume, shares outstanding and the market trend? \\
	\end{enumerate}


%Problem
\prob One of the most intriguing asset pricing anomalies in finance is so-called \textquotedblleft price momentum effect\textquotedblright that was covered in this chapter. This exercise using CRSP data is meant to guide you through creating a momentum strategy. The goal is to replicate the price momentum strategy described in Jegadeesh and Titman (2001)~\cite{JeTi} using more recent data extending to December 2014. [This problem is based on the notes from Ravi Jagannathan.] \\

\textbf{A. Source of Data}
The monthly stock returns from January 1965 to December 2014 can be obtained from the web. You should construct the decile ranking by yourself. For future reference, the decile ranking file is available from Ken French's web site. The NYSE market capitalization decile contains all common shares regardless of the end of the month price.\footnote{The Fama-French factors can be downloaded from Ken French's website. Note that you should only use the data from \textquotedblleft U.S. Research Returns Data\textquotedblright\ section.}You should use each stock's permanent identification code (PERMNO) as a unique identifier for that stock.

Note that not all stocks have valid monthly returns for various reasons. One reason is that the stock gets delisted somewhere during the month. For the construction of your momentum portfolios, you can simply ignore the delisting returns in your portfolio construction and return accumulations for the sake of simplicity. More specifically, for this exercise, if the end of the month return of any stock is missing, you may simply assume the return is zero in your compounding. You should keep track of such cases (identify the reason for delisting/missing observation) and prepare a table summarizing how many such cases occurred. \\

\textbf{B. Stocks Selection Criterion}
Consider all the common stocks traded on AMEX, NYSE and NASDAQ but
subject to the following filtering rules:
	\begin{itemize}
	\item Exclude all stocks priced lower than $\$5$ at the end of month ($t$).
	\item Exclude all stocks belonging to the smallest NYSE market decile at the end of month ($t$). You should first find the cutoff for the NYSE smallest market decile, then delete all the stocks (including stocks on AMEX, NYSE, and NASDAQ) with a market cap that's below the cutoff.
	\item Another thing to take in consideration is that when you form your portfolios, you should only apply the filters to the stocks when you form the portfolio (i.e. at the end of month $t$). The future returns (stock returns at month $t+2$ to $t+7$) should not be filtered. Otherwise it creates a look-ahead bias. For example, for portfolio returns at month $t+2$ to $t+7$, you drop stocks with a price $<$5 at any month between $t+2$ to $t+7$. First, when forming the portfolio at time t, you use information you don't have at month t. Second, you eliminate stocks with likely poor performance (low price) in the future. The portfolio returns (especially the loser portfolio returns) are likely to be biased upward from this action. \\
\end{itemize}


Again, you should keep track of such cases and prepare a table summarizing how many such cases occurred. \\

\textbf{C. Portfolio Formation}
At the end of each month ($t$), sort the stocks into $10$ decile portfolios based on the six month cumulative returns between month ($t-5$) to ($t$). The first decile contains the stocks with the lowest past six-month returns, and the tenth decile contains the stocks with the highest past six-month returns.
	\begin{itemize}
	\item Start by accumulating the return of each decile portfolio between ($%
t+2 $) to ($t+7$), i.e. skipping the first month following the portfolio
formation.
	\item At the beginning of each month, construct overlapping decile portfolios as equally weighted averages of the past six month months' decile portfolios. That is, at each month $\tau $, the $i$th overlapping decile portfolio consists of an equal weighted average of the\ $i$th decile portfolios of the previous six month ($\tau -5$) to ($\tau $).\footnote{These 10 overlapping decile portfolios are the testing assets in our asset
pricing test exercises.}
	\item For each month and of the 10 overlapping decile portfolios, record the portfolio returns between January 1995 and December 2014. \\
	\end{itemize}

\textbf{D. Questions of Interests}
After constructing the relevant portfolios, answer the following questions. When you answer these questions, note any additional assumptions/choices you made. \\


\textbf{Momentum strategy profits.}
	\begin{enumerate}[a.)]
	\item Report the monthly average returns of your overlapping decile portfolio, as well as the winner ($P1$, highest past six month return portfolios) minus losers ($P10$, lowest past six month return portfolios).
	\item Report the monthly average returns of your overlapping decile portfolio for January and non-January months respectively, as well as the winner ($P1$, highest past six month return portfolios) minus losers ($P10$, lowest past six month return portfolios).
	\item Using CAPM and Fama-French version of multifactor asset pricing model to test the spreads between winner and loser portfolio ($P10$ - $P1$). \ The Fama-French factors can be obtained from Kent French's web site.
	\item Tabulate these results in a table similar to table $I$ in Jegadeesh and Titman (2001)~\cite{JeTi}.
	\item Compared your results to the results in Jegadeesh and Titman (2001)~\cite{JeTi} \emph{and comment on your findings.} \\
	\end{enumerate}

\textbf{Momentum strategy and market capitalization:}
	\begin{enumerate}
	\item[f.)] Repeat the above exercise but split the sample into small-cap and big-cap portfolio, where the small-cap portfolio contains the stocks in the lower five NYSE market capitalization deciles.
	\item[g.)] Tabulate these results in a table similar to table $I$ in Jegadeesh
and Titman (2001)~\cite{}.
	\item[h.)] Compare the simple momentum strategy's profits, and momentum strategy cut by the market capitalization profits to value-growth strategy's return. The value-growth strategy return can be proxied by the HML factor obtained from Ken French's website.
	\item[i.)] \emph{Comment on your findings}. In particular, briefly suggest at least three reasons why momentum profits may be related to market capitalization, and how would you go ahead and test it further.
	\end{enumerate}


%Problem
\prob This is another exercise using data for sixteen stocks to replicate the price momentum strategy albeit in a small scale.The daily returns from January 11, 2000 to November 20, 2005 for sixteen stocks are in the file \texttt{d\_logret\_16stocks};
to this add two more columns: SP500 returns and risk free rate.
    \textbf{Portfolio Formation:} \hfill \break
    On the last day of each month, sort the stocks into four quartile portfolios based on the past six month
cumulative returns. The first quartile contains the stocks with the lowest past six-month returns, and the
fourth quartile contains the stocks with the highest past six-month returns.
        \begin{enumerate}[(a)]
        \item Start by accumulating the return of each quartile portfolio for next six months
        \item For each month and of the four quartile portfolios, record the average portfolio returns between January 2002 and November 2005.
        \end{enumerate}
    \textbf{Questions of Interests} \hfill \break
    Same as in Exercise~5. \\


%Problem
\prob Consider the 30~min price bar data for Treasury Yields from June 8, 2006 to August 29, 2013. Daily data for Treasury Yields is just a snapshot at 4~P.M. and the total volume for the day is the sum of 30~min volumes since the previous day. This is the same data used in Section~\ref{sec:illus_ex} in the text.
   \begin{enumerate}[(a)]
   \item Develop ARMA time series models for both daily and the 30~min volumes. We can use daily data
for developing macro strategies and 30~min data for micro strategies.
\item Compare the exponential smoothing approach suggested in the text with ARMA modeled in (a).
   \item Compute volatility measure, $\sigma_t^2 = 0.5[\ln{\frac{H_t}{L_t}}]^2 - 0.386[\ln{\frac{C_t}{O_t}}]^2$ for both daily and 30~min levels and compare them
   \item Compute volatility forecasts, $\hat{\sigma}_{t+1}^2$ = weighted average of $\sigma_{t}^2, \sigma_{t-1}^2,...,\sigma_{t-22}^2$, for daily data and $\hat{\sigma}_{t+1\cdot m}^2$ = weighted average of $\sigma_{t\cdot m}^2,...,\sigma_{t-22\cdot m}^2$ for $m^{th}$ 30-min interval. [ In (a), you should notice strong seasonality in the 30-min data]
   \item Develop trading strategies that incorporate
      \begin{enumerate}
      \item price information only
      \item price and volume information and
      \item price, volume and volatility information
      \end{enumerate}
         \item Demonstrate that volume and volatility add value to the enter/exist strategies.
   \end{enumerate}




 


\addtocounter{total}{\theproblem}